\documentclass[../main.tex]{subfiles}

\begin{document}
\chapter{Differenzierbarkeit}
\section{Differenzierbare Funktionen}
\begin{goal}
  Formalisiere
  folgende Aussage:
  ``der Funktionsgraph einer Funktion
  $f \colon \mathbb{R} \to \mathbb{R}$
  hat im Punkt $(p, f(p))$ eine 
  wohldefinierte Tangente''.
\end{goal}

Affine Funktionen 
\begin{align*}
  A \colon \mathbb{R} & \to \mathbb{R} \\
  x & \mapsto ax + b
\end{align*}
für $a, b \in \mathbb{R}$,
siehe Abbildung~\ref{fig:affin},
sind der Prototyp
einer solchen Funktion.
Für alle $p \in \mathbb{R}$ und
alle $h \in \mathbb{R}$ gilt
\[
  A(p+h) = a(p+ h) + b = A(p) + L(h),
\]
wobei
\begin{align*}
  L \colon \mathbb{R} & \to \mathbb{R} \\
  h & \mapsto ah
\end{align*}
linear ist (im Gegensatz
zu der Funktion
$A \colon \mathbb{R} \to \mathbb{R}$,
welche ``nur'' affin ist).
Wir erhalten, dass für festes $p$ 
gilt, dass
  $A(p+h) = A(p) + L(h)$
die Summe einer konstanten
Funktion mit einer linearen Funktion ist.
Eine solche Aufteilung ist
eine zu restriktive Bedingung
für allgemeinere Funktionen,
da diese nur für affine Funktionen
gilt.


\begin{figure}[htb]
  \centering
  \includestandalone{images/affine}
  \caption{Der
  Graph einer affinen Funktion
$x \mapsto ax + b$}
  \label{fig:affin}
\end{figure}

\subsection*{Dreigliedentwicklung}
\begin{definition}
  Seien $a, b \in \mathbb{R}$ 
  mit $a < b$ und
  $f \colon (a, b) \to \mathbb{R}$ 
  eine Funktion.
  Weiterhin sei $p \in (a, b)$ und
  $d > 0$ 
  mit $(p - d, p + d) \subset (a, b)$.
  Die Funktion $f \colon (a, b) \to \mathbb{R}$ 
  heisst im Punkt
  $p \in (a, b)$ \emph{differenzierbar},
  falls eine lineare Funktion
  ${(Df)}_p \colon\mathbb{R} \to \mathbb{R}$
  und eine Funktion ${(Rf)}_p \colon (-d, d) \to \mathbb{R}$ 
  existieren, die folgende Eigenschaften erfüllen.
  \begin{enumerate}[(i)]
    \item Für alle $h \in (-d, d)$ gilt
      \[
        f(p + h) = f(p) + {(Df)}_p(h) + {(Rf)}_p(h).
      \]
    \item Für alle $\varepsilon > 0$ existiert
      $\delta > 0$ mit $\delta \leq d$, so dass
      für alle $h \in (- \delta, \delta)$ gilt,
      dass
      \[
        |{(Rf)}_p(h)| \leq \varepsilon \cdot |h|.
      \]
   \end{enumerate}
   Die Aufteilung
   \[
     f(p + h) = f(p) + {(Df)}_p(h) + {(Rf)}_p(h)
   \]
   heisst \emph{Dreigliedentwicklung} von $f$ 
   an der Stelle $p$.
   Die lineare Abbildung
   ${(Df)}_p \colon\mathbb{R} \to \mathbb{R}$ heisst
   \emph{Differential} von $f$ an der Stelle $p$,
   und ${(Rf)}_p$ heisst der \emph{Restterm}.
\end{definition}

Für die Bedingung, dass für
alle $\varepsilon > 0$ ein $\delta > 0$ 
mit $\delta \leq d$ existiert,
so dass für $h \in (-\delta, \delta)$ gilt,
dass
\[
  |{(Rf)}_p(h)| \leq \varepsilon \cdot |h|,
\]
schreiben wir häufig kurz
\[
  \lim_{|h| \to 0} \frac{|{(Rf)}_p(h)}{|h|} = 0.
\]

\begin{examples}
  \leavevmode
  \begin{enumerate}[(1)]
    \item Sei $L \colon \mathbb{R} \to \mathbb{R}$ 
      linear.
      Dann gilt für alle $p \in \mathbb{R}$ und alle
      $h \in \mathbb{R}$, dass
      \[
        L(p + h) = L(p) + L(h) + 0.
      \]
      Setze ${(DL)}_p = L \colon \mathbb{R} \to \mathbb{R}$.
      Diese Funktion ist linear. Setze
      ${(RL)}_p(h) = 0$. Diese Funktion erfüllt
      \[
        \lim_{h \to 0} \frac{|{(RL)}_p(h)|}{|h|} = 0.
      \]
      Also ist $L \colon \mathbb{R} \to \mathbb{R}$
      an der Stelle $p \in \mathbb{R}$ 
      differenzierbar mit Differential
      \[
      {(DL)}_p = L \colon \mathbb{R} \to \mathbb{R}.
      \]
    \item Betrachte die Abbildung
      \begin{align*}
        f \colon \mathbb{R} & \to \mathbb{R} \\
        x & \mapsto x^2.
      \end{align*}
      Es gilt für alle $p \in \mathbb{R}$ und
      $h \in \mathbb{R}$, dass
      \[
        f( p + h ) = {(p + h)}^2
        = p^2 + 2ph + h^2
        = f(p) + 2ph + h^2.
      \]
      Setze
      \[
        {(Df)}_p(h) = 2ph
      \]
      und
      \[
        {(Rf)}_p(h) = h^2.
      \]
      Die Funktion
      \begin{align*}
        {(Df)}_p \colon \mathbb{R} & \to \mathbb{R} \\
        h & \mapsto 2ph
      \end{align*}
      ist linear (für festes $p$).
      Die Funktion
      \begin{align*}
        {(Rf)}_p \colon \mathbb{R} & \to \mathbb{R} \\
        h & \mapsto h^2
      \end{align*}
      erfüllt
      \[
        \lim_{h \to 0} \frac{|{(Rf)}_p(h)|}{|h|} = 0.
      \]
      Tatsächlich, sei $\varepsilon > 0$ vorgegeben.
      Setze $\delta = \varepsilon > 0$.
      Für alle $h \in \mathbb{R}$ mit $|h| \leq \delta$ gilt,
      dass
      \[
        |{(Rf)}_p(h)| = |h| \cdot |h| \leq \varepsilon \cdot |h|.
      \]
      Also ist $f$ im Punkt $p \in \mathbb{R}$ differenzierbar
      mit Differential
      \[
        {(Df)}_p(h) = 2p \cdot h.
      \]
  \end{enumerate}
\end{examples}


\end{document}
