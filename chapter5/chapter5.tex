\documentclass[../main.tex]{subfiles}

\begin{document}
\chapter{Das Riemannsche Integral}
Das Riemannsche Integrals wurde
vom Schüler Bernhard Riemann (1826--1866)
von Gauss eingeführt.

Sei $f \colon \mathbb{R} \to \mathbb{R}$ differenzierbar
mit $f' = 0$ (die Nullfunktion).
Dann ist $f$ konstant (was wir bereits mithilfe des
Mittelwertsatz gezeigt haben). Daraus können wir
folgendes schliessen.
Seien $f, g \colon \mathbb{R} \to \mathbb{R}$ 
Funktionen mit
$f' = g'$, das heisst, für alle $x \in \mathbb{R}$
gilt $f'(x) = g'(x)$.
Dann ist $f - g$ konstant.
Wir folgern, dass die Funktion $f \colon \mathbb{R} \to \mathbb{R}$ 
durch ihre Ableitung $f' \colon \mathbb{R} \to \mathbb{R}$ 
und einen Wert $f(0)$ determiniert ist.

\begin{question}
  Wie bestimmen wir $f$ aus $f'$ 
  und $f(0)$?
\end{question}

\begin{specialcase}
  Sei $f$ affin. Dann gilt für alle
  $x \in \mathbb{R}$, dass
  \[
    f(x) = f(0) + x \cdot f'(0).
  \]
  Dies gilt im allgemeinen nicht: wir verwenden
  hier, dass $f$ affin ist.
  Aber immerhin ist für kleine $|h|$
  der Ausdruck $f(x) + h \cdot f'(x)$ eine
  gute Approximation von $f(x + h)$.
\end{specialcase}

Dies führt zu der Überlegung, dass für beliebige
Funktionen der Ausdruck
\[
  f(x) = f(0) + (f(x/N) - f(0))
  + (f(2x/N) - f(x/N)) + \cdot
  + (f(x) - f((N-1)x/N))
\]
zu einer Approximation von $f$ aus
$f'$ führen könnte. Für grosse $N$ 
sollte nämlich
\[
  f(x) - f(0) \approx \sum_{k = 0}^{N - 1} 
  \frac{x}{N} \cdot f'(k \cdot x / N)
\]
eine gute Approximation sein, wobei
hier das Symbol $\approx$ für ein
saloppes ``ungefähr'' steht.
Im Grenzübergang erhalten wir das
Integral
\[
  \int_{0}^{x} f'(t) \, dt =
  \lim_{N \to \infty} \sum_{k = 0}^{N - 1} \frac{x}{N}
  f'(k \cdot x/N).
\]
Nun definieren wir dieses Integral formal.

\section{Die Definition des Riemannschen Integrals}\label{sec:riemann-definition}
\begin{definition}
  Seien $a,b \in \mathbb{R}$
  mit $a < b$. Eine \emph{Partition} des Intervalls
  $[a, b]$ ist eine Unterteilung in
  endlich viele Teilintervalle der
  Form
  $I_k = [x_k, x_{k  + 1}]$ mit
  \[
    a = x_0 < x_1 < \cdots < x_n = b.
  \]
  Wir notieren diese Partition als
  $P = [x_0, x_1, \dots, x_n]$.
\end{definition}

Sei nun $f \colon [a, b] \to \mathbb{R}$ von oben und
unten beschränkt und $P$ eine Partition von $[a, b]$.
Definiere die korrespondierende \emph{Obersumme} durch
\[
  \overline S_P(f)
  = \sum_{k = 0}^{n - 1} (x_{k + 1} - x_k)
  \cdot \sup \left\{f(x) \mid x \in I_k \right\}
\]
und die korrespondierende \emph{Untersumme} durch
\[
  \underline S_P(f)
  = \sum_{k = 0}^{n - 1} (x_{k + 1} - x_k)
  \cdot \inf \left\{f(x) \mid x \in I_k \right\},
\]
siehe Abbildung~\ref{fig:riemann}.

\begin{figure}[htb] 
  \centering
  \begin{minipage}{0.50\textwidth}
    \centering

    \includestandalone{images/sum-upper}
  \end{minipage}%
  \begin{minipage}{0.50\textwidth}
    \centering

    \includestandalone{images/sum-lower}
  \end{minipage}%
  \caption{Obersumme und Untersumme}%
  \label{fig:riemann}
\end{figure}

\begin{lemma*}
  Seien $P_1$ und $P_2$ Partitionen
  von $[a, b]$. Dann gilt
  $\underline S_{P_1} \leq \overline S_{P_2}$.
  In anderen Worten sind alle Untersummen
  kleiner als alle Obersummen.
\end{lemma*}

\begin{proof}
  Schreibe
  $P_1 = [x_0, x_1, \dots, x_n]$ 
  und
  $P_2 = [y_0, y_1, \dots, y_m]$.
  Wähle eine Partition $Q$ von
  $[a,b]$, welche alle $x_i$ und
  $y_j$ enthält, eine sogenannte
  \emph{gemeinsame Verfeinerung}
  von $P_1$ und $P_2$.
  Dann gilt
  \[
    \underline S_{P_1} \leq \underline S_Q
    \leq \overline S_Q
    \leq \overline S_{P_2}.
  \]
  Dass die Untersummen beim Verfeinern
  grösser und die Obersummen kleiner
  werden, ist eine kleine Übung.
\end{proof}

\begin{definition}
  Eine beschränkte Funktion
  $f \colon [a, b] \to \mathbb{R}$ heisst
  \emph{Riemann-integrierbar}, falls
  \[
    \sup_{P} \underline S_P(f) = \inf_{Q} \overline S_Q(f)
  \]
  gilt,
  wobei $P$ und $Q$ alle Partitionen von
  $[a, b]$ durchlaufen.
  Diese Zahl heisst dann \emph{Riemann-Integral}
  von $f$ über $[a,b]$ und wird
  mit
  \[
    \int_{a}^{b} f(x) \, dx
  \]
  notiert.
\end{definition}

\begin{examples}
  \leavevmode
  \begin{enumerate}[(1)]
    \item Betrachte die Funktion
      \begin{align*}
        f \colon [0, 1] & \to \mathbb{R} \\
        x & \mapsto 
        \begin{cases}
          1 & \text{falls $x \in \mathbb{Q}$},\\
          0 & \text{falls $x \in \mathbb{R} \setminus \mathbb{Q}$}.
        \end{cases}
      \end{align*}
      Für alle Partitionen $P$ von $[0, 1]$ gilt
      \[
        \overline S_P(f) = 1
      \]
      und
      \[
        \underline S_P(f) = 0.
      \]
      Also ist $f$ nicht Riemann-integrierbar.
    \item Betrachte die Funktion
      \begin{align*}
        f \colon [-1, 1] & \to \mathbb{R} \\
        x & \mapsto 
        \begin{cases}
          0 & \text{falls $x \leq 0$},\\
          1 & \text{falls $x > 0$}.
        \end{cases}
      \end{align*}
      Betrachte die Partition
      $P = [-1, 0, 1/N, 1]$.
      Es gilt
      \[
        \overline S_P(f) = 1
      \]
      und
      \[
        \underline S_P(f) = 1 - 1/N.
      \] 
      Somit folgt $\overline S_P - \underline S_P = 1/N$,
      also ist $f$ Riemann-integrierbar und es gilt
      \[
        \int_{0}^{1} f(x) \, dx = 1.
      \]
  \end{enumerate}
\end{examples}

\begin{theorem}\label{thm:continuous-integrable}
  Sei $f \colon [a, b] \to \mathbb{R}$ stetig. Dann ist $f$ 
  Riemann-integrierbar.
\end{theorem}

\begin{proof}
  Da $[a, b]$ kompakt ist, impliziert die Stetigkeit von $f$ sogar
  \emph{gleichmässige Stetigkeit} auf $[0, 1]$:
  Für alle $\varepsilon > 0$ existiert $\delta > 0$ so, dass
  für alle $p, q \in [a, b]$ mit $|q - p| \leq \delta$ gilt, dass
  $|f(q) - f(p)| \leq \varepsilon$.
  Wähle $N \in \mathbb{N}$ mit
  \[
    \frac{b-a}{N} \leq \delta.
  \]
  Betrachte die Partition
   \[
    P = \left[ a, a + \frac{b - a}{N}, a + 2 \frac{b-a}{N}, \dots,
    b\right].
  \]
  Für diese gilt
  \begin{align*}
    \overline S_P(f) - \underline S_P(f)
    &\leq \sum_{k=0}^{N-1} \frac{b-a}{N}\cdot \varepsilon  \\
    & N \frac{b-a}{\varepsilon} \\
    &= \varepsilon \cdot (b-a).
  \end{align*}
  Wir folgern, dass die Differenz $\overline S_P(f) - \underline S_P(f)$ 
  beliebig kleine Werte annehmen kann. Insbesondere gilt,
  dass
  \[
    \lim_{N \to \infty} \overline S_P(f) - \underline S_P(f) = 0.
  \]
  Folglich ist $f$ Riemann-integrierbar.
\end{proof}

\begin{example}
  Betrachte die Funktion
  \begin{align*}
    f \colon [0, 1] & \to \mathbb{R} \\
    x & \mapsto x^n
  \end{align*}
  für $n \in \mathbb{N}$ beliebig. Betrachte die
  Partition
  \[
    P = [0, 1/N, 2/N, \dots, 1].
  \]
  Es gilt:
  \[
    \overline S_P(f) = \frac{1}{N} \cdot \left( {\left( \frac{1}{N}  \right)}^n
    + \cdots + {\left( \frac{N}{N}  \right)}^n \right)
  \]
  und
  \[
    \underline S_P(f) = \frac{1}{N} \cdot \left( 0 +{\left( \frac{1}{N}  \right)}^n 
    + \cdots + {\left( \frac{N-1}{N}  \right)}^n\right).
  \]
  Also folgt
  \[
    \lim_{N \to \infty} \overline S_P(f) - \underline S_P(f) = 
    \lim_{N \to \infty}\frac{1}{N} = 0,
  \]
  was wir auch aus Theorem~\ref{thm:continuous-integrable} schliessen konnten.
  Wir wissen aber immer noch nicht, was das Integral
  \[
    I = \int_{0}^{1} x^n \, dx
  \]
  für einen Wert hat.
\end{example}

\begin{geometric}
  Das Integral
  \[
    \int_{a}^{b} f(x) \, dx
  \]
  ist der Flächeninhalt zwischen der $x$-Achse und dessen
  Funktionsgraph zwischen $a$ und $b$ mit Gewichtung
  wie in Abbildung~\ref{fig:riemann-integral}.
  Wir werden dies als Definition des Wortes
  \textit{Flächeninhalt} auffassen.
\end{geometric}

\begin{figure}[htb]
  \centering
  \includestandalone{images/riemann-integral}
  \caption{Die Vorzeichen der Flächen}%
  \label{fig:riemann-integral}
\end{figure}

\begin{remark}
  Stetigkeit in allen Punkten von $[a, b]$ ist
  nicht notwendig: Es gibt unstetige Funktionen,
  die Riemann-integrierbar sind.
\end{remark}

\begin{examples}
  \leavevmode
  \begin{enumerate}[(i)]
    \item Die Funktion
      \begin{align*}
        f \colon [-1, 1] & \to \mathbb{R} \\
        x & \mapsto 
        \begin{cases}
          0 & x \leq 0,\\
          1 & x > 0.
        \end{cases}
      \end{align*}
      ist nicht stetig, aber dennoch Riemann-integrierbar.
    \item Die Funktion
      \begin{align*}
        f \colon [0, 1] & \to \mathbb{R} \\
        x & \mapsto 
        \begin{cases}
          0 & x = 0 \\
          1/x & x > 0
        \end{cases}
      \end{align*}
      ist unbeschränkt und deshalb nicht Riemann-integrierbar.
      Für alle Partitionen $P$ von $[0, 1]$ gilt
      $
      \overline S_P(f) = + \infty$.
      Dieses Phänomen ist der Grund dafür, dass
      wir Beschränktheit von $f$ fordern.
  \end{enumerate}
\end{examples}

\subsection*{Integrabilitätskriterium von Lebesgue}
Henri Lebesgue (1875--1941) war ein Französischer Mathematiker.
Sein Integrabilitätskriterium charakterisiert die
Riemann-integrierbaren Funktionen, im Gegensatz zu
Theorem~\ref{thm:continuous-integrable}, was nur eine
hinreichende Bedingung darstellt.

\begin{definition}
  Eine Teilmenge $\Delta \subset \mathbb{R}$ 
  heisst \emph{Lebesgue Nullmenge},
  falls für alle $\varepsilon > 0$ abzählbar viele
  offene Intervalle
  \[
    I_k = (a_k, b_k)
  \]
  mit $k \in \mathbb{N}$
  existieren, so dass 
  \[
    \Delta \subset \bigcup_{k=0}^{\infty} I_k
  \]
  und
  \[
    \sum_{k=0}^{\infty} b_k - a_k \leq \varepsilon.
  \]
\end{definition}

\begin{example}
  Sei $\Delta = \mathbb{Q} \subset \mathbb{R}$.
  Sei $\varepsilon > 0$ vorgegeben.
  Wähle eine Bijektion $q \colon\mathbb{N} \to \mathbb{Q}$ 
  und schreibe $q_k = q(k)$ für $k \in \mathbb{N}$.
  Setze
  \[
    a_k = q_k - \frac{\varepsilon}{4} \cdot \frac{1}{2^k}
  \]
  und
  \[
    b_k = q_k + \frac{\varepsilon}{4} \cdot \frac{1}{2^k}.
  \]
  Es gilt $q_k \in (a_k, b_k)$ und  
  \[
    b_k - a_k = \frac{\varepsilon}{2^{k+1}}.
  \]
  Es folgt, dass
  \[
    \sum_{k=0}^{\infty} b_k - a_k = \varepsilon \cdot
    \sum_{k=0}^{\infty} \frac{1}{2^{k+1}} = \varepsilon.
  \]
  Wir haben also $\mathbb{Q}$ abgedeckt mit Intervallen,
  deren Gesamtlänge kleiner als $\varepsilon$ ist.
  Also ist $\mathbb{Q}$ (und, mit dem selben Beweis,
  jede abzählbare Teilmenge von $\mathbb{R}$)
  eine Lebesgue Nullmenge.
\end{example}

\begin{theorem}[Lebesgue 1901]
  Sei $f \colon [a, b] \to \mathbb{R}$ eine Funktion.
  Dann sind folgende Aussagen äquivalent.
  \begin{enumerate}[\normalfont(i)]
    \item $f$ ist  Riemann-integrierbar,
    \item $f$ ist beschränkt und die Menge
      $\Delta$ der Unstetigkeitsstellen von $f$ ist
      eine Lebesgue Nullmenge.
  \end{enumerate}
\end{theorem}

\begin{proof}
  Siehe zum Beispiel Abschnitt 84 in~\cite{heuser}.
\end{proof}

In anderen Vorlesungen, zum Beispiel Analysis 3, wird
der Begriff der Lebesgue-Integrabilität diskutiert.
Mit dem Lebesgue-Integral sind sogar noch mehr Funktionen
integrierbar.

\begin{example}
  Die Funktion
  \begin{align*}
    f \colon [0, 1] & \to \mathbb{R} \\
    x & \mapsto 
    \begin{cases}
      1/q & \text{falls $x = p/q \in \mathbb{Q}$ gekürzt ist},\\
      0 & \text{falls $x \in \mathbb{R} \setminus Q$}
    \end{cases}
  \end{align*}
  ist beschränkt
  und auf $\mathbb{R} \setminus \mathbb{Q}$ stetig,
  also Riemann-integrierbar. In den Übungen wird das
  ``zu Fuss'' mit der Definition des Riemann-Integrals
  gezeigt.
\end{example}

\subsection*{Elementare Eigenschaften des Riemann-Integrals}
\begin{notation}
  Die Menge $R[a, b]$ ist definiert als
  die Menge aller Riemann-integrierbaren Funktionen
  $f \colon [a, b] \to \mathbb{R}$.
\end{notation}

Wir stellen fest:
\begin{enumerate}[(i)]
  \item $R[a, b]$ ist ein reeller Vektorraum.
  \item die Abbildung
    \begin{align*}
      \int \colon R[a,b] & \to \mathbb{R} \\
      f & \mapsto \int_{a}^{b} f(x) \, dx
    \end{align*}
    ist linear.
\end{enumerate}
Konkret heisst das folgendes.
\begin{enumerate}[(i)]
  \item Seien $f, g \colon [a, b] \to \mathbb{R}$
    Riemann-integrierbar und $\lambda \in \mathbb{R}$.
    Dann ist $f + \lambda g \colon [a, b] \to \mathbb{R}$ 
    auch Riemann-integrierbar.
  \item Die Gleichung
    \[
      \int_{a}^{b} f(x) + \lambda g(x) \, dx
      = \int_{a}^{b} f(x) \, dx
      + \lambda \cdot \int_{a}^{b} g(x) \, dx
    \]
    ist für alle Riemann-integrierbaren 
    $f, g \colon [a, b] \to \mathbb{R}$ 
    erfüllt.
\end{enumerate}

\begin{remark}
  Der Vektorraum $R[a,b]$ ist unendlichdimensional, da die
  Funktionen 
  \[
   x \mapsto x^k 
  \]
  linear unabhängig sind.
\end{remark}

An dieser Stelle halten wir ausserdem fest, dass
für alle $c \in [a, b]$ und alle Riemann-integrierbaren
$f \colon [a, b] \to \mathbb{R}$ gilt, dass
\[
  \int_{a}^{b} f(x) \, dx = \int_{a}^{c} f(x) \, dx
  + \int_{c}^{b} f(x) \, dx.
\]

\section{Der Hauptsatz der Differential- und Integralrechnung}
\begin{definition}
  Sei $f \colon [a, b] \to \mathbb{R}$ eine Funktion.
  Eine differenzierbare
  Funktion $F \colon [a, b] \to \mathbb{R}$ ist
  eine \emph{Stammfunktion} von $f$, falls für
  alle $x \in [a, b]$ gilt, dass $F'(x) = f(x)$.
  Hier einigen wir uns auf die Konvention,
  dass Differenzierbarkeit im Punkt $a$ bedeutet, dass
  der Grenzwert
  \[
    F'(a) 
    = \lim_{h \to 0} \frac{F(a + |h|) - F(a)}{|h|} \in \mathbb{R}
  \]
  existiert. Ähnlich heisst Differenzierbarkeit im Punkt $b$,
  dass der Grenzwert
  \[
    F'(b) = \lim_{h \to 0} \frac{F(b - |h|) - F(b)}{-|h|}
  \]
  existiert.
\end{definition}


\begin{theorem}\label{thm:fundamental}
  Sei $f \colon [a, b] \to \mathbb{R}$ stetig.
  Dann ist die Funktion
  \begin{align*}
    F \colon [a, b] & \to \mathbb{R} \\
    x & \mapsto \int_{a}^{x} f(t) \, dt
  \end{align*}
  eine Stammfunktion von $f$.
\end{theorem}

\begin{proof}
  Sei $p \in [a, b]$ und $h \in \mathbb{R}$ 
  mit $p + h \in [a, b]$.
  Wir leiten nun eine Dreigliedentwicklung
  für $F$ bei $p$ her. Berechne
  \begin{align*}
    F(p+h)
    &= \int_{a}^{p+h} f(t) \, dt\\
    &= \int_{a}^{p} f(t) \, dt
    + \int_{p}^{p+h} f(t) \, dt.
  \end{align*}
  Schreibe
  \[
    \int_{p}^{p+h} f(t) \, dt
    = \int_{p}^{p+h} f(p) \, dt
    + \int_{p}^{p+h} f(t) - f(p) \, dt.
  \]
  Bemerke, dass
  \[
    \int_{p}^{p+h} f(p) \, dt = h \cdot f(p).
  \]
  Setzen wir
  \[
    {(RF)}_p(h) = \int_{p}^{p+h} f(t) - f(p) \, dt,
  \]
  so gilt
  \[
    F(p + h) = F(p) + h \cdot f(p) + {(RF)}_p(h).
  \]
  Wir zeigen nun, dass ${(RF)}_p(h)$ relativ klein in $|h|$ ist.
  Sei dazu $\varepsilon > 0$ vorgegebn.
  Wähle $\delta > 0$ so, dass für alle
  $t \in [a, b]$ mit $|t - p| \leq \delta$ gilt,
  dass $|f(t) - f(p)| \leq \varepsilon$.
  Dann gilt für $h \in \mathbb{R}$ mit $|h| \leq \delta$,
  dass
  \begin{align*}
    |{(RF)}_p(h)|
    & = \left| \int_{p}^{p+h} f(t) - f(p) \, dt \right|\\
    & \leq |h| \cdot \varepsilon,
  \end{align*}
  da für alle Obersummen gilt, da alle Partitionen
  die Bedingungen
  $\overline S_P(f|_{[p, p+h]}) \leq |h| \cdot \varepsilon$
  und $\underline S_P(f|_{[p, p+h]}) \geq - \varepsilon \cdot h$ 
  erfüllen.
  Wir schliessen, dass der Restterm
  ${(RF)}_p(h)$ relativ klein in $|h|$ ist.
  Also ist $F$ differenzierbar im Punkt $p$ mit
  \[
    {(DF)}_p(h) = h \cdot f(p),
  \]
  also gilt $F'(p) = f(p)$.
\end{proof}

\begin{application}
  Sei $U \subset \mathbb{R}$ offen und $g \colon U \to \mathbb{R}$ 
  stetig differenzierbar. Sei $[a, b] \subset U$.
  Dann gilt
  \[
    \int_{a}^{b} g'(t) \, dt = g(b) - g(a).
  \]
\end{application}

\begin{proof}
  Betrachte die Funktion
  \begin{align*}
    G \colon [a, b] & \to \mathbb{R} \\
    x & \mapsto \int_{a}^{x} g'(t) \, dt.
  \end{align*}
  Nach Theorem~\ref{thm:fundamental} ist $G$ differenzierbar
  und $G'(x) = g'(x)$, das heisst $G - g$ ist auf
  $[a, b]$ konstant (nach dem Mittelwertsatz).
  Es existiert somit $c \in \mathbb{R}$ mit $G(x) = g(x) + c$.
  Setzen wir $x = a $ ein, so erhalten wir
  \[
    G(a) = \int_{a}^{a} g'(t) \, dt = 0,
  \]
  also gilt $c = -g(a)$. Wir schliessen, dass
  \[
    \int_{a}^{b} g(t) \, dx = G(b) = g(b) - g(a). \qedhere
  \]
\end{proof}

\begin{example}
  Es gilt
  \[
    \int_{a}^{b} x^n \, dx = \frac{1}{n+1} (b^{n+1} - a^{n+1})
  \]
  für $n \in \mathbb{N}$, da die Funktion
  \begin{align*}
    G \colon \mathbb{R} & \to \mathbb{R} \\
    x & \mapsto \frac{x^{n+1}}{n+1}
  \end{align*}
  eine Stammfunktion von
  \begin{align*}
    g \colon \mathbb{R} & \to \mathbb{R} \\
    x & \mapsto x^n
  \end{align*}
  ist. Der Spezialfall
  \[
    \int_{0}^{1} x^n \, dx = \frac{1}{n+1}
  \]
  beschreibt nun das Integral, das wir nach
  dem Beweis von Theorem~\ref{thm:continuous-integrable}
  noch nicht berechnen konnten.
\end{example}


\end{document}
