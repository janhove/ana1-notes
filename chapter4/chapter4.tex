\documentclass[../main.tex]{subfiles}

\begin{document}
\chapter{Differenzierbarkeit}
\section{Differenzierbare Funktionen}
\begin{goal}
  Formalisiere
  folgende Aussage:
  ``der Funktionsgraph einer Funktion
  $f \colon \mathbb{R} \to \mathbb{R}$
  hat im Punkt $(p, f(p))$ eine 
  wohldefinierte Tangente''.
\end{goal}

Affine Funktionen 
\begin{align*}
  A \colon \mathbb{R} & \to \mathbb{R} \\
  x & \mapsto ax + b
\end{align*}
für $a, b \in \mathbb{R}$,
siehe Abbildung~\ref{fig:affin},
sind der Prototyp
einer solchen Funktion.
Für alle $p \in \mathbb{R}$ und
alle $h \in \mathbb{R}$ gilt
\[
  A(p+h) = a(p+ h) + b = A(p) + L(h),
\]
wobei
\begin{align*}
  L \colon \mathbb{R} & \to \mathbb{R} \\
  h & \mapsto ah
\end{align*}
linear ist (im Gegensatz
zu der Funktion
$A \colon \mathbb{R} \to \mathbb{R}$,
welche ``nur'' affin ist).
Wir erhalten, dass für festes $p$ 
gilt, dass
  $A(p+h) = A(p) + L(h)$
die Summe einer konstanten
Funktion mit einer linearen Funktion ist.
Eine solche Aufteilung ist
eine zu restriktive Bedingung
für allgemeinere Funktionen,
da diese nur für affine Funktionen
gilt.


\begin{figure}[htb]
  \centering
  \includestandalone{images/affine}
  \caption{Der
  Graph einer affinen Funktion
$x \mapsto ax + b$}%
  \label{fig:affin}
\end{figure}

\subsection*{Dreigliedentwicklung}
\begin{definition}
  Seien $a, b \in \mathbb{R}$ 
  mit $a < b$ und
  $f \colon (a, b) \to \mathbb{R}$ 
  eine Funktion.
  Weiterhin sei $p \in (a, b)$ und
  $d > 0$ 
  mit $(p - d, p + d) \subset (a, b)$.
  Die Funktion $f \colon (a, b) \to \mathbb{R}$ 
  heisst im Punkt
  $p \in (a, b)$ \emph{differenzierbar},
  falls eine lineare Funktion
  ${(Df)}_p \colon\mathbb{R} \to \mathbb{R}$
  und eine Funktion ${(Rf)}_p \colon (-d, d) \to \mathbb{R}$ 
  existieren, die folgende Eigenschaften erfüllen.
  \begin{enumerate}[(i)]
    \item Für alle $h \in (-d, d)$ gilt
      \[
        f(p + h) = f(p) + {(Df)}_p(h) + {(Rf)}_p(h).
      \]
    \item Für alle $\varepsilon > 0$ existiert
      $\delta > 0$ mit $\delta \leq d$, so dass
      für alle $h \in (- \delta, \delta)$ gilt,
      dass
      \[
        |{(Rf)}_p(h)| \leq \varepsilon \cdot |h|.
      \]
   \end{enumerate}
   Die Aufteilung
   \[
     f(p + h) = f(p) + {(Df)}_p(h) + {(Rf)}_p(h)
   \]
   heisst \emph{Dreigliedentwicklung} von $f$ 
   an der Stelle $p$.
   Die lineare Abbildung
   ${(Df)}_p \colon\mathbb{R} \to \mathbb{R}$ heisst
   \emph{Differential} von $f$ an der Stelle $p$,
   und ${(Rf)}_p$ heisst der \emph{Restterm}.
\end{definition}

Für die Bedingung, dass für
alle $\varepsilon > 0$ ein $\delta > 0$ 
mit $\delta \leq d$ existiert,
so dass für $h \in (-\delta, \delta)$ gilt,
dass
\[
  |{(Rf)}_p(h)| \leq \varepsilon \cdot |h|,
\]
schreiben wir häufig kurz
\[
  \lim_{|h| \to 0} \frac{|{(Rf)}_p(h)}{|h|} = 0.
\]
Wir sagen auch, dass ${(Rf)}_p \colon (-d, d) \to \mathbb{R}$ \emph{relativ klein}
in $|h|$ ist.

\begin{examples}
  \leavevmode
  \begin{enumerate}[(1)]
    \item Sei $L \colon \mathbb{R} \to \mathbb{R}$ 
      linear.
      Dann gilt für alle $p \in \mathbb{R}$ und alle
      $h \in \mathbb{R}$, dass
      \[
        L(p + h) = L(p) + L(h) + 0.
      \]
      Setze ${(DL)}_p = L \colon \mathbb{R} \to \mathbb{R}$.
      Diese Funktion ist linear. Setze
      ${(RL)}_p(h) = 0$. Diese Funktion erfüllt
      \[
        \lim_{h \to 0} \frac{|{(RL)}_p(h)|}{|h|} = 0.
      \]
      Also ist $L \colon \mathbb{R} \to \mathbb{R}$
      an der Stelle $p \in \mathbb{R}$ 
      differenzierbar mit Differential
      \[
      {(DL)}_p = L \colon \mathbb{R} \to \mathbb{R}.
      \]
    \item Betrachte die Abbildung
      \begin{align*}
        f \colon \mathbb{R} & \to \mathbb{R} \\
        x & \mapsto x^2.
      \end{align*}
      Es gilt für alle $p \in \mathbb{R}$ und
      $h \in \mathbb{R}$, dass
      \[
        f( p + h ) = {(p + h)}^2
        = p^2 + 2ph + h^2
        = f(p) + 2ph + h^2.
      \]
      Setze
      \[
        {(Df)}_p(h) = 2ph
      \]
      und
      \[
        {(Rf)}_p(h) = h^2.
      \]
      Die Funktion
      \begin{align*}
        {(Df)}_p \colon \mathbb{R} & \to \mathbb{R} \\
        h & \mapsto 2ph
      \end{align*}
      ist linear (für festes $p$).
      Die Funktion
      \begin{align*}
        {(Rf)}_p \colon \mathbb{R} & \to \mathbb{R} \\
        h & \mapsto h^2
      \end{align*}
      erfüllt
      \[
        \lim_{h \to 0} \frac{|{(Rf)}_p(h)|}{|h|} = 0.
      \]
      Tatsächlich, sei $\varepsilon > 0$ vorgegeben.
      Setze $\delta = \varepsilon > 0$.
      Für alle $h \in \mathbb{R}$ mit $|h| \leq \delta$ gilt,
      dass
      \[
        |{(Rf)}_p(h)| = |h| \cdot |h| \leq \varepsilon \cdot |h|.
      \]
      Also ist $f$ im Punkt $p \in \mathbb{R}$ differenzierbar
      mit Differential
      \[
        {(Df)}_p(h) = 2p \cdot h.
      \]
  \end{enumerate}
\end{examples}

\subsection*{Eindeutigkeit der Dreigliedentwicklung}
Seien
\[
  f(p + h) = f(p) + {(Df)}_p(h) + {(Rf)}_p(h)
  = f(p) + \widetilde{{(Df)}}_p(h) + \widetilde{{(Rf)}}_p(h)
\]
zwei Dreigliedentwicklungen von $f \colon (a, b) \to \mathbb{R}$.
Dann ist die Funktion
\[
{(Df)}_p - \widetilde{{(Df)}}_p \colon \mathbb{R} \to \mathbb{R}
\]
linear und die Funktion
$
  {(Rf)}_p - \widetilde{{(Rf)}}_p
$
relativ klein in $|h|$.
Weiter gilt auf dem gemeinsamen Definitionsbereich der
beiden Funktionen, dass
\[
{(Df)}_p - \widetilde{{(Df)}}_p 
  = {(Rf)}_p - \widetilde{{(Rf)}}_p.
\]


\begin{lemma*}
  Sei $g \colon \mathbb{R} \to \mathbb{R}$ linear und
  relativ klein in $|h|$.
  Dann ist $g$ die Nullfunktion.
\end{lemma*}

\begin{proof}
  Die Funktion $g$ ist linear, das heisst, für alle
  $h \in \mathbb{R}$ gilt
  \[
    g(h) = h \cdot g(1).
  \]
  Weiter ist $g$ relativ klein in $|h|$.
  Sei $\varepsilon > 0$. Dann existiert
  $\delta > 0$ so dass für $|h| \leq \delta$ 
  gilt, dass
  \[
   |g(1)| \cdot |h| = |h \cdot g(1) | = |g(h)| \leq \varepsilon \cdot |h|,
  \]
  also $|g(1)| \leq \varepsilon$.
  Da $\varepsilon > 0$ beliebig war, folgt
  $g(1) = 0$ und somit $g(h) = h \cdot g(1) = 0$.
  In anderen Worten ist $g$ die Nullfunktion.
\end{proof}

\begin{corollary}
  Die Dreigliedentwicklung ist eindeutig.
\end{corollary}

\begin{proof}
Seien
\[
  f(p + h) = f(p) + {(Df)}_p(h) + {(Rf)}_p(h)
  = f(p) + \widetilde{{(Df)}}_p(h) + \widetilde{{(Rf)}}_p(h)
\]
zwei Dreigliedentwicklungen von $f \colon (a, b) \to \mathbb{R}$.
Wir haben oben gesehen, dass
\[
  g = 
{(Df)}_p - \widetilde{{(Df)}}_p 
  = {(Rf)}_p - \widetilde{{(Rf)}}_p.
\]
linear und relativ klein in $|h|$, also
die Nullfunktion ist.
\end{proof}

\begin{application}
  Sei $f \colon \mathbb{R} \to \mathbb{R}$ \emph{symmetrisch},
  das heisst, $f(x) = f(-x)$ für alle $x \in \mathbb{R}$,
  und im Punkt $0$ differenzierbar.
  Dann ist ${(Df)}_0 = 0$ die Nullfunktion.

  Tatsächlich, sei
  \[
    f(0 + h) = f(0) + {(Df)}_0(h) + {(Rf)}_0(h)
  \]
  die Dreigliedentwicklung von $f$ bei $p = 0$.
  Da $f$ symmetrisch ist, ist auch
  \[
    f(0 + h) = f(0 - h) = f(0) + {(Df)}_0(-h) + {(Rf)}_0(-h).
  \]
  Mit ${(Df)}_0(-h) = - {(Df)}_0(h)$ und Eindeutigkeit
  der Dreigliedentwicklung folgt, dass
  \[
    {(Df)}_0 = - {(Df)}_0(h),
  \]
  also ${(Df)}_0(h) = 0$ für alle $h$.
\end{application}


\begin{example}
  Betrachte die Funktion
  \begin{align*}
    f \colon \mathbb{R} & \to \mathbb{R} \\
    x & \mapsto |x|,
  \end{align*}
  siehe Abbildung~\ref{fig:modulus}.
  Dann ist $f$ symmetrisch. Wir behaupten, dass
  $f$ im Punkt $p = 0$ nicht differenzierbar ist.
  Falls doch, ist muss ${(Df)}_0 = 0$ nach obiger
  Anwendung gelten, also
  Folgt
  \[
    |h| = f(0 + h) = f(0) + {(Df)}_0(h) + {(Rf)}_0(h)
    = {(Rf)}_0(h).
  \]
  Aber die Funktion
  \[
    {(Rf)}_0(h) = |h|
  \]
  ist nicht relativ klein in $|h|$. Tatsächlich existiert
  für $\varepsilon < 1$ kein $h \neq 0$ mit
  \[
    |{(Rf)}_p(h)| \leq \varepsilon \cdot |h|.
  \]
  Das ist ein Widerspruch, also ist $f$ im Nullpunkt
  nicht differenzierbar.
\end{example}

\begin{figure}[htb]
  \centering
  \includestandalone{images/modulus}
  \caption{Der Graph der Funktion $x \mapsto |x|$}%
  \label{fig:modulus}
\end{figure}

\begin{examples}
  \leavevmode
  \begin{enumerate}[(1)]
    \item Sei $f(x) = c$ für eine
      Konstante $c \in \mathbb{R}$.
      Dann gilt für alle $p \in \mathbb{R}$ 
      und $h \in \mathbb{R}$, dass
      \[
        f(p + h) = f(p) + 0 + 0,
      \]
      also ist $f$ differenzierbar und
      ${(Df)}_p = 0$ ist die Nullfunktion.
    \item Sei $f(x) = x^n$ für $n \in \mathbb{N}$
      mit $n \geq 2$. Die Fälle $n = 0$ und $n = 1$ 
      haben wir mit dem konstanten und dem linearen
      Fall bereits behandelt.
      Für alle $p \in \mathbb{R}$ und
      $h \in \mathbb{R}$ gilt
      \[
        f(p + h) = {(p + h)}^n
        = p^n + np^{n-1}h + h^2 \sum_{k=2}^{n} \binom{n}{k}
        p^{n-k}h^{k-2}.
      \]
      Setze nun
      \[
        {(Df)}_p(h) = np^{n-1}h
      \]
      und
      \[
        {(Rf)}_p(h) = h^2 \sum_{k=2}^{n} \binom{n}{k}.
      \]
      Dann ist ${(Df)}_p$ linear in $h$.
      Wir behaupten, dass ${(Rf)}_p(h)$ relativ
      klein in $|h|$ ist.
      Sei dazu $\varepsilon > 0$ vorgegeben.
      Setze
      \[
        M = \sum_{k=2}^{n} \binom{n}{k}|p|^{n-k} \geq 0
      \]
      und
      \[
        \delta = \min \{1, \varepsilon / (M + 1)\}.
      \]
      Dann gilt für alle $h \in \mathbb{R}$ 
      mit $|h| \leq \delta$,
      dass
      \[
        |{(Rf)}_p(h)| = |h|^2 \cdot \sum_{k=2}^{n}  p^{n-k}h^{k-2}
                 = |h|^2 \cdot M
                 \leq |h| \cdot \delta \cdot M 
                 \leq|h| \cdot  \varepsilon.
      \]
      Also ist ${(Rf)}_p(h)$ relativ klein in $|h|$ 
      und somit ist $f$ im Punkt $p$ differenzierbar und
      \[
        {(Df)}_p(h) = n p^{n-1} \cdot h.
      \]
    \item Sei $f(x) = e^x$. Dann ist
      \[
        f(p + h) = e^{p+h} = e^p \cdot e^h
        = e^p \cdot (1  +h + h^2/2! + \cdots)
      \]
      Setze
      \[
        {(Df)}_p(h) = e^p \cdot h
      \]
      und
      \[
        {(Rf)}_p(h) = e^p \cdot (h^2/2! + h^3/3! + \cdots).
      \]
      Dann ist
      \[
        e^{p+h} = e^p + {(Df)}_p(h) + {(Rf)}_p(h)
      \]
      und ${(Df)}_p$ ist linear.
      Dass ${(Rf)}_p(h)$ relativ klein
      in $|h|$ ist, ist eine Aufgabe auf Serie~6.
    \item Sei $f(x) = \log(x)$, definiert auf
      $\mathbb{R}_{>0}$.
      Es ist schwierig, die Dreigliedentwicklung
      von $f$ aufzuschreiben.
    \item Sei $f(x) = \sqrt x$, definiert auf
      $\mathbb{R}_{\geq 0}$.
      Auch für diese Funktion $f$ ist es schwierig,
      die Dreigliedentwicklung aufzuschreiben.
  \end{enumerate}
\end{examples}

Wir werden später sehen, unter welchen Bedingungen
die Umkehrfunktion einer bijektiven
differenzierbaren Funktion differenzierbar ist.

\begin{remarks}
  \leavevmode
  \begin{itemize}
\item
  Für $f(x) = x^n$ und $p \in \mathbb{R}$ beliebig
  gilt
  \[
    {(Df)}_p(h) = n p^{n-1} \cdot h.
  \]
\item
  Für $f(x) = e^x$ und $p \in \mathbb{R}$ beliebig gilt
  \[
    {(Df)}_p(h) = e^p \cdot h.
  \]
  \end{itemize}
  Diese Grössen vor dem $h$ erkennen wir
  wohl wieder.
\end{remarks}

\begin{definition}
  Sei $f \colon (a, b) \to \mathbb{R}$ differenzierbar
  im Punkt $p \in (a, b)$. Dann heisst
  \[
    f'(p) = {(Df)}_p(1)
  \]
  die \emph{Ableitung} von $f$ an der Stelle $p$.
\end{definition}

\begin{proposition*}
  Sei $f \colon (a, b) \to \mathbb{R}$ eine Funktion
  und $p \in (a, b)$ beliebig.
  Dann sind folgende Aussagen äquivalent.
  \begin{enumerate}[\normalfont(i)]
    \item $f$ ist an der Stelle $p$ differenzierbar.
    \item Der Grenzwert
      \[
        \lim_{h \to 0} \frac{f(p+h) - f(p)}{h}
      \]
      existiert.
  \end{enumerate}
  Falls diese Aussagen zutreffen, gilt
  \[
    f'(p) = \lim_{h \to 0} \frac{f(p + h)}{h}.
  \]
\end{proposition*}

\begin{remark}
  Wieso verwenden wir nicht den Punkt (ii) der
  Proposition als Definition der Differenzierbarkeit?
  Das hat einen relativ einfachen Grund:
In höheren Dimensionen ist der Ausdruck
$(f(p + h) - f(p))/h$
nicht definiert.
\end{remark}

\begin{example}
  Betrachte die Spiegelung
  \begin{align*}
    f \colon \mathbb{R}^2 & \to \mathbb{R}^2 \\
    (x, y) & \mapsto (x, -y)
  \end{align*}
  an der $x$-Achse, siehe
  Abbildung~\ref{fig:reflection}. Dann ist $f$ linear.

\begin{figure}[htb]
  \centering
  \includestandalone{images/reflection}
  \caption{Spiegelung an der $x$-Achse}%
  \label{fig:reflection}
\end{figure}

  Versuche,
$(f(p + h) - f(p))/h$
auszuwerten. Da $f$ linear ist, folgt
\[
  f(p + h) = f(p) + f(h).
\]
Daraus folgt, 
dass
\[
  \frac{f(p+h) - f(p)}{h} = \frac{f(h)}{h}
  = \frac{(h_1, -h_2)}{(h_1, h_2)}.
\]
Aber das Verhältnis zweier linear unabhängiger
Vektoren ist nicht definiert,
das heisst der Punkt (ii) in der Proposition
macht keinen Sinn.
Die Funktion $f$ ist aber sehr wohl
nach unserer Definition differenzierbar,
da sie linear ist. Die Dreigliedentwicklung
ist
\[
  f(p + h) = f(p) + f(h) + 0,
\]
also ist ${(Df)}_p = f$.

In der Analysis 2, wenn wir Funktionen
auf höherdimensionalen Räumen
studieren, werden wir also auf
die Dreigliedentwicklung angewiesen sein.
Weiter werden sich alle Beweise, die wir
hier über die Dreigliedentwicklung führen,
auf diese kompliziertere Situation
übertragen lassen.
\end{example}


\begin{proof}[Beweis der Proposition]
  Für die Implikation ``(i) $\Rightarrow$ (ii)''
  sei
  \[
    f(p + h) = f(p) + {(Df)}_p(h) + {(Rf)}_p(h)
  \]
  die Dreigliedentwicklung von $f$ an der
  Stelle $p$.
  Für $h \neq 0$ gilt, dass
  \[
    \frac{f(p + h) - f(p)}{h}
    = \frac{{(Df)}_p(h) + {(Rf)}_p(h)}{h}
    =(Df)_p(1) + \frac{{(Rf)}_p(h)}{h}.
  \]
  Da ${(Rf)}_p(h)$ relativ klein in $|h|$ ist,
  folgt dass
  \[
    \lim_{h \to 0} \frac{|{(Rf)}_p(h)}{|h|} = 0,
  \]
  also auch
  \[
    \lim_{h \to 0} \frac{{(Rf)}_p(h)}{h} = 0.
  \]
  Wir folgern, dass
  \[
    \lim_{h \to 0} \frac{f(p + h) - f(p)}{h}
    = f'(p) + \lim_{h \to 0} \frac{{(Rf)}_p(h)}{h} = f'(p).
  \]
  Insbesondere existiert dieser Grenzwert.

  Wir beweisen nun die Implikation
  ``(ii) $\Rightarrow$ (i)''.
  Nehme an, der Grenzwert
  \[
    a = \lim_{h \to 0} \frac{f(p+h) - f(p)}{h} \in \mathbb{R}
  \]
  existiert.
  Im Hinblick auf $a = f'(p) = {(Df)}_p(1)$.
  machen wir den Ansatz
  \[
    f(p + h) = f(p) + a \cdot h + {(Rf)}_p(h).
  \]
  für eine Dreigliedentwicklung von $f$ an der Stelle
  $p \in (a, b)$. Zu zeigen ist, dass
  ${(Rf)}_p(h)$ relativ klein in $|h|$ ist.
  Für $h \neq 0$ gilt
  \[
    \frac{{(Rf)}_p(h)}{h} =
    \frac{f(p+h) - f(p) - ah}{h}
    = \frac{f(p+h) - f(h)}{h} - a.
  \]
  Im Grenzwert gilt also
  \[
    \lim_{h \to 0} \frac{{(Rf)}_p(h)}{h}
    = \lim_{h \to 0}
    \frac{f(p + h) - f(p)}{h} - a
    = a - a = 0.
  \]
  Also folgt auch
  \[
    \lim_{h \to 0} \frac{|{(Rf)}_p(h)|}{|h|} = 0,
  \]
  und somit ist ${(Rf)}_p(h)$ relativ klein
  in $|h|$.
  Wir folgern, dass ${(Df)}_p(h) = a \cdot h$,
  also insbesondere haben wir
  \[
    f'(p) = {(Df)}_p(1) = a = 
    \lim_{h \to 0} \frac{f(p + h) - f(p)}{h}. \qedhere
  \]
\end{proof}

\begin{geometric}
  Die Grösse
  \[
    \frac{f(p+h) - f(p)}{h}
  \]
  ist die Steigung der Sekante
  zwischen den Punkten $(p, f(p))$ 
  und $(p + h, f(p+h))$ im Graph
  der Funktion $f$, siehe Abbildung~\ref{fig:derivative}
  links.
  Lassen wir $h$ gegen null gehen,
  so erhalten wir im Grenzwert die Tangente
  mit Steigung
  \[
    f'(p) = \lim_{h \to 0} \frac{f(p + h) - f(p)}{h}.
  \]
\end{geometric}

\begin{figure}[htb] 
  \centering
  \begin{minipage}{0.4\textwidth}
    \centering
    \includestandalone{images/derivative1}
  \end{minipage}%
  \begin{minipage}{0.4\textwidth}
    \centering
    \includestandalone{images/derivative2}
  \end{minipage}%
  \caption{Sekante und Tangente}%
  \label{fig:derivative}
\end{figure}

\begin{examples}
  \leavevmode
  \begin{enumerate}[(1)]
    \item Sei $f(x) = \sqrt x$,
      definiert auf $\mathbb{R}_{\geq 0}$,
      siehe
      Wir versuchen, die Ableitung
      von $f$ an der Stelle $0$ auszurechnen.
      Berechne
      \[
        \lim_{h \to 0} \frac{f(0 + h) - f(0)}{h}
        = \lim_{h \to 0} \frac{\sqrt h}{h},
      \]
      aber dieser Grenzwert existiert nicht: 
      die Tangente der Funktion im Nullpunkt
      ist vertikal, hat also unendliche Steigung.
      Wir folgern, dass es keine Funktion
      \[
      \overline f \colon (- \delta, \delta) \to \mathbb{R}
      \]
      mit folgenden Eigenschaften gibt.
      \begin{enumerate}[(i)]
        \item $\overline f$ ist differenzierbar
          an der Stelle $p = 0$.
        \item Für alle $x \in [0, \delta]$ gilt
          $\overline f (x) = f(x)$.
      \end{enumerate}
      Allgemeiner würde man Differenzierbarkeit auf
      einem Randpunkt auch über eine solche
      Erweiterung $\overline f$ definieren.
      
      \begin{figure}[htb]
        \centering
        \includestandalone{images/sqrt}
        \caption{Die Wurzelfunktion $f(x) = \sqrt x$}%
        \label{fig:sqrt}
      \end{figure}

    \item Wir definieren geometrisch die Sinusfunktion
      und untersuchen diese auf Differenzierbarkeit.
      In einem Kreis mit Radius $1$, sei $x$ ein Winkel,
      das heisst, eine Strecke zwischen zwei Punkten auf
      dem Kreis.
      Wir definieren den \emph{Sinus},
      den \emph{Cosinus} und den \emph{Tangens} durch
      \[
        \sin(x) = s \text{, } \cos(x) = c \text{ und }
        \tan(x) = t,
      \]
      wobei $s$, $c$ und $t$ via Abbildung~\ref{fig:sine}
      definiert sind.
      Sowohl der Sinus als auch der Cosinus
      sind also beschränkte Funktionen.
      Es gilt, dass die Verhältnisse
      \[
        c : 1 = s : t
      \]
      übereinstimmen, das heisst
      \[
        t = \frac{s}{c}.
      \]
      Wir bemerken für
      $0 < x < \pi/2$, dass
      \[
        \cos(x) \leq \frac{\sin(x)}{x} \leq 1.
      \]
      Dies folgt aus der geometrischen Tatsache,
      dass $s \leq x \leq t$,
      wovon wir uns vom Bild überzeugen lassen.
      Also folgt
      \[
        \sin(x) \leq x \leq \frac{\sin(x)}{\cos(x)}.
      \]
      Weiter folgt
      \[
        \lim_{x \to 0} \frac{\sin(x)}{x} = 1,
      \]
      da 
      \[
        \lim_{x \to 0} \cos(x) = 1.
      \]
      Wir folgern, dass
      \[
        \sin'(0) = 
        \lim_{x \to 0} \frac{\sin(x) - \sin(0)}{x}
        = 1.
      \]
      Das heisst die Funktion 
      $\sin \colon \mathbb{R} \to \mathbb{R}$ ist differenzierbar
      bei $p = 0$.
      Aus den Additionstheoremen folgt leicht,
      dass die Sinusfunktion überall differenzierbar ist.

      \begin{figure}[htb]
        \centering
        \includestandalone{images/sine}
        \caption{Sinus und Cosinus}%
        \label{fig:sine}
      \end{figure}
      
  \end{enumerate}
  
\end{examples}




\end{document}
