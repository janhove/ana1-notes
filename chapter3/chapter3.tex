\documentclass[../main.tex]{subfiles}

\begin{document}
\chapter{Stetige Funktionen}

Folgende Definition von Weierstrass
ist aus gutem Grund sehr berühmt.

\begin{definition}
  Sei $A \subset \mathbb{R}$ 
  eine beliebige Teilmenge.
  Eine Funktion
  $f \colon A \to \mathbb{R}$ 
  heisst \emph{stetig} im Punkt
  $p \in A$, falls für
  alle vorgegebenen $\varepsilon > 0$
  ein  $\delta > 0$ existiert,
  so dass für alle $q \in A$ 
  mit $|q-p| \leq \delta$ gilt,
  dass $|f(q) - f(p)| \leq \varepsilon$.
  Falls $f \colon A \to \mathbb{R}$ in
  allen Punkten $p \in A$ stetig ist,
  dann heisst $f$ \emph{stetig}.
\end{definition}

\begin{examples}
  \leavevmode
  \begin{enumerate}[(1)]
    \item Die Identitätsfunktion
      \begin{align*}
        f \colon \mathbb{R} & \to \mathbb{R} \\
        p & \mapsto p
      \end{align*}
      ist stetig. Sei dazu $p \in \mathbb{R}$ fest
      und $\varepsilon > 0$ vorgegeben.
      Setze $\delta = \varepsilon$. Dann gilt
      für alle $q \in \mathbb{R}$ mit
      $|q - p| \leq \delta$, dass
      $|f(q) - f(p)| = |q - p| \leq \delta = \varepsilon$.
  \end{enumerate}
\end{examples}


\end{document}
