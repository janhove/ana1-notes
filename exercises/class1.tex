\documentclass[12pt,a4paper]{article}

\usepackage{tikz}
\usepackage{amsmath}
\usepackage{amsfonts}
\usepackage{amsthm}
\usepackage{amsmath}
\usepackage[german]{babel}
\usepackage{enumerate}

\newtheorem*{claim}{Behauptung}
\newtheorem*{lemma}{Lemma}

\theoremstyle{definition}
\newtheorem*{definition}{Definition}
\newtheorem*{examples}{Beispiele}
\newtheorem*{reminder}{Erinnerung}

\title{Übungsstunde 1}
\author{Analysis 1}
\date{25. September 2020}
\begin{document}
\maketitle

\section{Besprechung Serie 1}
\subsection*{Catalanzahlen}
\begin{reminder}
  Sei $n \in \mathbb N$. Die $n$-te Catalanzahl ist
  \[C_{n} = \frac{1}{n+1}\binom{2n}{n}.\]
\end{reminder}

\begin{lemma}
  Sei $n \in \mathbb N$. Dann gilt
  \[\frac{C_{n+1}}{C_{n}} = \frac{4n+2}{n+2}.\]
\end{lemma}

\begin{proof}
  Berechne
  \begin{align*}
    \frac{C_{n+1}}{C_{n}}
    &= \frac{n+1}{n+2} \left. \binom{2n + 2}{n + 1} \middle/ \binom{2n}{n} \right. \\
    &= \frac{n+1}{n+2} \cdot \frac{(2n+2)!{(n!)}^{2}}{{((n+1)!)}^{2}(2n)!} \\
    &= \frac{n+1}{n+2} \cdot \frac{(2n+2)(2n+1)}{{(n+1)}^{2}} \\
    &= \frac{4n^{2} + 6n + 2}{(n+1)(n+2)} \\
    &= \frac{(n+1)(4n+2)}{(n+1)(n+2)} \\
    &= \frac{4n+2}{n+2}. & \qedhere
  \end{align*}
\end{proof}


\end{document}
