\documentclass[../main.tex]{subfiles}

\begin{document}
\chapter{Das Riemannsche Integral}
Das Riemannsche Integrals wurde
vom Schüler Bernhard Riemann (1826--1866)
von Gauss eingeführt.

Sei $f \colon \mathbb{R} \to \mathbb{R}$ differenzierbar
mit $f' = 0$ (die Nullfunktion).
Dann ist $f$ konstant (was wir bereits mithilfe des
Mittelwertsatz gezeigt haben). Daraus können wir
folgendes schliessen.
Seien $f, g \colon \mathbb{R} \to \mathbb{R}$ 
Funktionen mit
$f' = g'$, das heisst, für alle $x \in \mathbb{R}$
gilt $f'(x) = g'(x)$.
Dann ist $f - g$ konstant.
Wir folgern, dass die Funktion $f \colon \mathbb{R} \to \mathbb{R}$ 
durch ihre Ableitung $f' \colon \mathbb{R} \to \mathbb{R}$ 
und einen Wert $f(0)$ determiniert ist.

\begin{question}
  Wie bestimmen wir $f$ aus $f'$ 
  und $f(0)$?
\end{question}

\begin{specialcase}
  Sei $f$ affin. Dann gilt für alle
  $x \in \mathbb{R}$, dass
  \[
    f(x) = f(0) + x \cdot f'(0).
  \]
  Dies gilt im allgemeinen nicht: wir verwenden
  hier, dass $f$ affin ist.
  Aber immerhin ist für kleine $|h|$
  der Ausdruck $f(x) + h \cdot f'(x)$ eine
  gute Approximation von $f(x + h)$.
\end{specialcase}

Dies führt zu der Überlegung, dass für beliebige
Funktionen der Ausdruck
\[
  f(x) = f(0) + (f(x/N) - f(0))
  + (f(2x/N) - f(x/N)) + \cdot
  + (f(x) - f((N-1)x/N))
\]
zu einer Approximation von $f$ aus
$f'$ führen könnte. Für grosse $N$ 
sollte nämlich
\[
  f(x) - f(0) \approx \sum_{k = 0}^{N - 1} 
  \frac{x}{N} \cdot f'(k \cdot x / N)
\]
eine gute Approximation sein, wobei
hier das Symbol $\approx$ für ein
saloppes ``ungefähr'' steht.
Im Grenzübergang erhalten wir das
Integral
\[
  \int_{0}^{x} f'(t) \, dt =
  \lim_{N \to \infty} \sum_{k = 0}^{N - 1} \frac{x}{N}
  f'(k \cdot x/N).
\]
Nun definieren wir dieses Integral formal.

\section{Die Definition des Riemannschen Integrals}
\begin{definition}
  Seien $a,b \in \mathbb{R}$
  mit $a < b$. Eine \emph{Partition} des Intervalls
  $[a, b]$ ist eine Unterteilung in
  endlich viele Teilintervalle der
  Form
  $I_k = [x_k, x_{k  + 1}]$ mit
  \[
    a = x_0 < x_1 < \cdots < x_n = b.
  \]
\end{definition}

Sei nun $f \colon [a, b] \to \mathbb{R}$ von oben und
unten beschränkt und $P$ eine Partition von $[a, b]$.
Definiere die korrespondierende \emph{Obersumme} durch
\[
  \overline S_P(f)
  = \sum_{k = 0}^{n - 1} (x_{k + 1} - x_k)
  \cdot \sup \left\{f(x) \mid x \in I_k \right\}
\]
und die korrespondierende \emph{Untersumme} durch
\[
  \underline S_P(f)
  = \sum_{k = 0}^{n - 1} (x_{k + 1} - x_k)
  \cdot \inf \left\{f(x) \mid x \in I_k \right\},
\]
siehe Abbildung~\ref{fig:riemann}.

\begin{figure}[htb] 
  \centering
  \begin{minipage}{0.50\textwidth}
    \centering

    \includestandalone{images/sum-upper}
  \end{minipage}%
  \begin{minipage}{0.50\textwidth}
    \centering

    \includestandalone{images/sum-lower}
  \end{minipage}%
  \caption{Obersumme und Untersumme}%
  \label{fig:riemann}
\end{figure}

\begin{lemma*}
  Seien $P_1$ und $P_2$ Partitionen
  von $[a, b]$. Dann gilt
  $\underline S_{P_1} \leq \overline S_{P_2}$.
  In anderen Worten sind alle Obersummen
  kleiner als alle Untersummen.
\end{lemma*}

\begin{proof}
  Schreibe
  $P_1 = [x_0, x_1, \dots, x_n]$ 
  und
  $P_2 = [y_0, y_1, \dots, y_m]$.
  Wähle eine Partition $Q$ von
  $[a,b]$, welche alle $x_i$ und
  $y_j$ enthält, eine sogenannte
  \emph{gemeinsame Verfeinerung}
  von $P_1$ und $P_2$.
  Dann gilt
  \[
    \underline S_{P_1} \leq \underline S_Q
    \leq \overline S_Q
    \leq \overline S_{P_2}.
  \]
  Dass die Untersummen beim Verfeinern
  grösser und die Obersummen kleiner
  werden, ist eine kleine Übung.
\end{proof}

\begin{definition}
  Eine beschränkte Funktion
  $f \colon [a, b] \to \mathbb{R}$ heisst
  \emph{Riemann-integrierbar}, falls
  \[
    \sup_{P} \underline S_P(f) = \inf_{Q} \overline S_Q(f)
  \]
  gilt,
  wobei $P$ und $Q$ alle Partitionen von
  $[a, b]$ durchlaufen.
  Diese Zahl heisst dann \emph{Riemann-Integral}
  von $f$ über $[a,b]$ und wird
  mit
  \[
    \int_{a}^{b} f(x) \, dx
  \]
  notiert.
\end{definition}

\begin{examples}
  \leavevmode
  \begin{enumerate}[(1)]
    \item Betrachte die Funktion
      \begin{align*}
        f \colon [0, 1] & \to \mathbb{R} \\
        x & \mapsto 
        \begin{cases}
          1 & \text{falls $x \in \mathbb{Q}$},\\
          0 & \text{falls $x \in \mathbb{R} \setminus \mathbb{Q}$}.
        \end{cases}
      \end{align*}
      Für alle Partitionen $P$ von $[0, 1]$ gilt
      \[
        \overline S_P(f) = 1
      \]
      und
      \[
        \underline S_P(f) = 0.
      \]
      Also ist $f$ nicht Riemann-integrierbar.
    \item Betrachte die Funktion
      \begin{align*}
        f \colon [-1, 1] & \to \mathbb{R} \\
        x & \mapsto 
        \begin{cases}
          0 & \text{falls $x \leq 0$},\\
          1 & \text{falls $x > 0$}.
        \end{cases}
      \end{align*}
      Betrachte die Partition
      $P = [-1, 0, 1/N, 1]$.
      Es gilt
      \[
        \overline S_P(f) = 1
      \]
      und
      \[
        \underline S_P(f) = 1 - 1/N.
      \] 
      Somit folgt $\overline S_P - \underline S_P = 1/N$,
      also ist $f$ Riemann-integrierbar und es gilt
      \[
        \int_{0}^{1} f(x) \, dx = 1.
      \]
      
      
      
      
      
      
      
  \end{enumerate}
  
\end{examples}




\end{document}
