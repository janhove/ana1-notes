\documentclass[../main.tex]{subfiles}

\begin{document}
\chapter{Folgen und Reihen}
\section{Folgen}
\begin{definition}
	Eine \emph{Folge} in $\mathbb{R}$ ist eine Abbildung
  \begin{align*}
    a \colon \mathbb{N} & \to \mathbb{R} \\
    n & \mapsto a_n.
  \end{align*}
  Wir notieren das häufig als 
  $(a_0, a_1, a_2, \dots) = {(a_n)}_{n \in \mathbb{N}}$.
\end{definition}

Die für uns wichtigste Eigenschaft von Folgen ist deren Konvergenz.

\begin{definition}
	Eine Folge ${(a_n)}_{n \in \mathbb{N}}$ in $\mathbb{R}$
	heisst \emph{konvergent} mit Grenzwert
	$L \in \mathbb{R}$, falls für alle
	$\varepsilon \in \mathbb{R}$ mit $\varepsilon > 0$
	eine Zahl $N \in \mathbb{N}$ existiert,
	so dass für alle $n\in \mathbb{N}$ 
	mit $n \geq N$ gilt, dass
	$|a_n - L| \leq \varepsilon$. In diesem Fall schreiben wir
	\[
	  \lim_{n \to \infty} a_n = L.
	\]
\end{definition}

In anderen Worten bedeutet $\lim_{n \to \infty} a_n = L$, dass
für vorgegebenes $\varepsilon > 0$
ab einem gewissen Index die Folge für immer im Intervall
$[L- \varepsilon, L + \varepsilon]$ liegt.

\begin{examples}
  \leavevmode
\begin{enumerate}[(1)]
  \item Sei $a_n = 1/n$ für $n \geq 1$. Wir behaupten,
	  dass diese Folge konvergent mit Grenzwert
	  $0 \in \mathbb{R}$ ist. Dazu sei $\varepsilon > 0$
	  vorgegeben.
	  Nach dem Archimedischen Prinzip
	  existiert $N \in \mathbb{N}$ mit
	  $N > 1/\varepsilon$.
	  Dann gilt für alle $n \in \mathbb{N}$ mit $n \geq N$,
	  dass $n > 1/\varepsilon$, also insbesondere
	  nach dem zweiten Ordnungsaxiom, dass
	  $1/n < \varepsilon$.
	  Folglich ist
	  \[
		  |a_n - 0| = \frac{1}{n} < \varepsilon,
	  \]
	  also
	  \[
		  \lim_{n \to \infty}\frac{1}{n} = 0.
	  \]
	  Varianten davon sind
	  \begin{itemize}
	    \item $\lim_{n \to \infty} \frac{1}{n^2} = 0$,
	    \item $\lim_{n \to \infty} \frac{1}{\sqrt n} = 0$,
	    \item $\lim_{n \to \infty} \frac{1}{n^a} = 0$ für
		    eine ``vernünftige'' Potenz $a$.
	  \end{itemize}
  \item Sei $a_n = \sqrt[n]{n}$. Wir behaupten, dass
	  \[
		  \lim_{n \to \infty} \sqrt[n]{n} = 1.
	  \]
	  Wir wollen zeigen, dass 
	  \[
		  \sqrt[n]{n} \leq 1 + \sqrt{\frac{2}{n-1}}.
	  \]
	  Tatsächlich gilt
	  \[
		  n  \leq {\left( 1 + \sqrt{\frac{2}{n-1}} \right)}^n,
	  \]
	  denn Anwenden der binomischen Formel
	  \[
		  {(a + b)}^n = \binom{n}{0}a^n + \binom{n}{1}a^{n-1}b
		  + \cdots + \binom{n}{n-1}ab^{n-1} + \binom{n}{n}b^n
	  \]
	  liefert
	  \[
		  {\left( 1 + \sqrt{\frac{2}{n-1}} \right)}^n
		  = 1 + n \cdot \sqrt{\frac{2}{n-1}}
		  + \binom{n}{2} {\left( \sqrt{\frac{2}{n-1}} \right)}^2
		  + R
	  \]
	  mit $R \geq 0$.
    Wir interessieren uns nur für den 
    quadratischen Term.
    Bemerke, dass
    \[
      \binom{n}{2}{\left( \sqrt{\frac{2}{n-1}} \right)}^2
      = \frac{n(n-1)}{2} \frac{2}{n-1} = n.
    \]
	  Somit gilt für alle $n \geq 1$ die Ungleichung
	  \[
		  1 \leq \sqrt[n]{n} \leq 1 + \sqrt{\frac{2}{n-1}}.
	  \]
	  Aus
	  \[
		  \lim_{n \to \infty} 1 + \sqrt{\frac{2}{n-1}} = 1,
	  \]
    was eine Variation vom ersten Beispiel ist folgt nun auch
	  \[
		  \lim_{n \to \infty} \sqrt[n]{n} = 1.
	  \]
  \item Sei $q \in \mathbb{R}$ mit $q \geq 0$. Dann gilt
	  \[
	    \lim_{n \to \infty} q^n = 
	    \begin{cases}
		    0, & \text{falls } q < 1, \\
		    1, & \text{falls } q = 1, \\
		    +\infty, & \text{falls } q > 1.
	    \end{cases}
	  \]
	Die Notation
	\[
	  \lim_{n \to \infty} a_n = + \infty
	\]
	heisst, dass für alle $S > 0$ ein Index $N \in \mathbb{N}$
	existiert, so dass für alle $n \geq N$ gilt, dass
	$a_n > S$ ist.
	Der zweite der Fälle ist klar. Für den dritten Fall, 
	betrachte
	\[
		q^n = {\left( 1 + (q-1) \right)}^n \geq 1 + n(q-1),
	\]
	wobei $q-1 > 0$. 
	Sei $S > 0$ vorgegeben. Wähle $N \in \mathbb{N}$ mit
	\[
	  N \geq \frac{S}{q-1}.
	\]
	Für alle $n \geq N$ gilt dann $q^n \geq 1 + S > S$.
	Für den ersten Fall ersetze $q$ durch $1/q$.
\end{enumerate}
\end{examples}

\section{Cauchyfolgen}
\begin{definition}
Eine Folge 
${\left( a_n \right)}_{n \in \mathbb{N}}$ in $\mathbb{R}$ 
heisst \emph{Cauchyfolge}, so dass für alle
vorgegebenen $\varepsilon$ ein $N \in \mathbb{N}$ existiert,
so dass für alle $n, m \in \mathbb{N}$ mit $n,m \geq N$ gilt,
dass $|a_n - a_m| \leq \varepsilon$.
\end{definition}

In anderen Worten, ab einem gewissen Index sind alle
Folgenglieder höchstens $\varepsilon$ voneinander 
entfernt. Die Definition der Konvergenz unterscheidet
sich dadurch, dass bei Cauchyfolgen kein Grenzwert
erwähnt wird: nur Folgeglieder werden verglichen.

\begin{manualtheorem}{1}[Konvergenzprinzip
  von Cauchy, 1789--1857]
Für Folgen ${\left( a_n \right)}_{n \in \mathbb{N}}$ in $\mathbb{R}$
  sind äquivalent:
  \begin{enumerate}[\normalfont(i)]
    \item Die Folge ${\left( a_n \right)}_{n \in \mathbb{N}}$
      ist konvergent.
    \item Die Folge ${\left( a_n \right)}_{n \in \mathbb{N}}$
      ist eine Cauchyfolge.
  \end{enumerate}
\end{manualtheorem}

\begin{proof}
  Die Implikation (i) $\Rightarrow$ (ii) ist eine Übung.
  Dazu kann  
  \[
    |a_n - a_m| \leq |a_n - L| + |L - a_m|
  \]
  verwendet werden. Diese Ungleichung wird auch
  \emph{Dreiecksungleichung} genannt (das Dreieck
  ist hier degeneriert, das heisst es liegt
  auf einer Linie, und die Eckpunkte sind
  $a_n, a_m$ und $L$).

  Die Umkehrung (ii) $\Rightarrow$ (i) ist etwas schwieriger.
Sei ${\left( a_n \right)}_{n \in \mathbb{N}}$ eine
  Cauchyfolge.
  Betrachte für alle $k \in \mathbb{N}$ die Menge
  \[
    A_k = \left\{ a_{k}, a_{k+1}, \dots \right\}
    \subset \mathbb{R}.
  \]
  Wir behaupten, dass diese Mengen $A_k$
  (von oben und unten)
  beschränkt sind.
  Zum Beweis betrachte $\varepsilon = 1 > 0$.
  Dann existiert ein $N \in \mathbb{N}$ mit
  der Eigenschaft, dass immer wenn
  $n, m \geq N$, dann folgt
  $|a_n - a_m| \leq 1$.
  Setze
  \[
  S_k = \max \{|a_k|, |a_{k+1}|, \dots, |a_{N-1}|, |a_N| + 1\}.
  \]
  Dann gilt für alle $n \geq k$, dass $|a_k| \leq S_k$.
  Tatsächlich ist für $n \geq N$ die Ungleichung
  $|a_n - a_N| \leq \varepsilon = 1$
  erfüllt, also folgt $|a_n| \leq |a_N| + 1$.
  Analog existiert eine untere Schranke $I_k$ für 
  $A_k$.
  Dies zeigt die Behauptung.

  Nun gilt für alle $k \in \mathbb{N}$, dass
  $I_k \leq A_k \leq S_k$ (wobei $A_k \subset \mathbb{R}$
  und $I_k, S_k \in \mathbb{R}$).
  Wende die Vollständigkeit von $\mathbb{R}$ 
  auf die beiden Teilmengen
  $\left\{ I_0, I_1, \dots \right\} $
  und $\{S_0, S_1, \dots\}$ an: Es existiert
  $L \in \mathbb{R}$, so dass für alle
  $k \in \mathbb{N}$ gilt, dass $I_k \leq L \leq S_k$.

  Wir behaupten nun, dass
  \[
    \lim_{n \to \infty} a_n = L.
  \]
  Um das zu beweisen, müssen wir die formale
  Definition der Konvergenz anwenden.
  Sei also $\varepsilon > 0$ vorgegeben.
  Dann existiert $N \in \mathbb{N}$ 
  so dass immer wenn $n, m \geq N$, dann auch
  $|a_n - a_m| \leq \varepsilon$
  (da ${(a_n)}_{n \in \mathbb{N}}$ eine Cauchyfolge ist).
  Daraus folgt, dass
  \[
    |\sup A_N - \inf A_N| \leq \varepsilon,
  \]
  da die nicht-strikte Ungleichung ja für alle
  Elemente von $A_N$ gilt.
  Nach Konstruktion gilt $\inf A_N \leq L \leq \sup A_N$.
  Ausserdem gilt für alle $n \geq N$, dass
  \[
    |a_n - L| \leq |S_N - I_N| \leq \varepsilon,
  \]
  da $I_N \leq a_n \leq S_N$.
\end{proof}

Zusammengefasst haben wir die Folgeglieder
erst ab einem späten Index betrachtet und
unter Verwendung der Vollständigkeit (!)
gezeigt, dass wir einen Grenzwert $L$
finden können.

\begin{remark}
  In $\mathbb{R}$ (und auch
  in einem allgemeineren Setting)
  sind folgende Prinzipien äquivalent.
  \begin{itemize}
    \item Vollständigkeit,
    \item das Supremumsprinzip,
    \item das Konvergenzprinzip von Cauchy.
  \end{itemize}
  
\end{remark}


Die Beobachtung, dass die Folge $I_0, I_1, I_2, \dots$
im Beweis oben konvergiert, lässt sich verallgemeinern.

\begin{claim}[Monotononieprinzip]
  Sei ${(a_n)}_{n \in \mathbb{N}}$ eine monoton wachsende
  Folge in $\mathbb{R}$ (das heisst, für alle
  $n \in \mathbb{N}$ gilt, dass $a_n \leq a_{n+1}$),
  welche nach oben beschränkt ist.
  Dann konvergiert ${(a_n)}_{n \in \mathbb{N}}$ mit
  Grenzwert
  \[
    S = \sup \{a_0, a_1, a_2, \dots\}.
  \]
\end{claim}

\begin{proof}
  Da $\{a_0, a_1, a_2, \dots\}$ nach oben beschränkt ist,
  ist $S \in \mathbb{R}$ definiert und eine reelle Zahl.
  Wir zeigen nun, dass
  \[
    \lim_{n \to \infty} a_n = S
  \]
  gilt.
  Sei dazu $\varepsilon > 0$ vorgegeben. 
  Dann ist $S - \varepsilon$ keine obere Schranke
  für $\{a_0, a_1, a_2, \dots\}$,
  da $S$ ja die kleinste obere Schranke ist.
  Also existiert $N \in \mathbb{N}$ mit
  $a_N > S - \varepsilon$.
  Dann gilt für alle $n \geq N$, dass
  \[
    S - \varepsilon < a_N \leq a_n \leq S,
  \]
  also $|a_n - S| \leq \varepsilon$.
\end{proof}

\begin{example}
  Betrachte die Folge ${(a_n)}_{n \in \mathbb{N}}$ 
  gegeben durch
   \[
   a_n = {\left( 1 + \frac{1}{n} \right)}^n.
   \]
   Es gilt $a_1 = 2$, $a_2 = 9/4$, $a_3 = 64/27, \dots$.
   Wir behaupten, dass
   \begin{enumerate}[(i)]
     \item ${(a_n)}_{n \in \mathbb{N}}$ monoton wachsend ist,
     \item für alle $n \in \mathbb{N}$ gilt, dass
       $a_n \leq 3$.
   \end{enumerate}
   Der Beweis dieser Behauptung ist eine Übung auf
   der Serie 4. Hier machen wir deshalb nur eine Beweisskizze:
   \begin{enumerate}[(i)]
     \item Es kann gezeigt werden, dass
       \[
         {\left( 1 + \frac{1}{n} \right) }^n \leq
         {\left( 1 + \frac{1}{n+1} \right)}^{n+1},
       \]
       indem 
       \begin{align*}
         {\left( 1 - \frac{1}{{(n+1)}^2} \right)}^n &=
         {\left( \frac{1 + \frac{1}{n+1}}{1 + \frac{1}{n}} \right) } \\& 
         \geq \frac{1}{1 + \frac{1}{n+1}}\\
                                                                                   &= 1 - \frac{1}{n+2}
       \end{align*}
       berechnet wird.
     \item
       Die Ungleichung ${(1 + 1/n)}^n \leq 3$ ist äquivalent
       zur Ungleichung ${(n + 1)}.^n \leq 3n^n$.
       Man kann dann zeigen, dass
       \[
         {(n+1)}^n =
         \sum_{k=0}^{n} \binom{n}{k}n^k 
         \leq \left( 1 + \frac{1}{1!}
         + \frac{1}{2!} + \frac{1}{3!} + \cdots +
       \frac{1}{n!}\right) \cdot n^n.
       \]
   \end{enumerate}
\end{example}

\begin{remark}
  Der Grenzwert der Folge oben ist
  \[
    \lim_{n \to \infty} {\left( 1 + \frac{1}{n} \right)}^n = e,
  \]
  die \emph{Eulersche Zahl} (die wir noch nicht definiert haben).
\end{remark}

\subsection*{Teilfolgen}
Wir bemerken, dass
\begin{enumerate}[(i)]
  \item jede konvergente Folge beschränkt ist,
  \item es beschränkte Folgen gibt, welche nicht konvergieren,
    zum Beispiel $a_n = {(-1)}^n$ (wieso?).
\end{enumerate}

\begin{definition}
  Sei ${(a_n)}_{n \in \mathbb{N}}$ eine Folge in $\mathbb{R}$ 
  und ${(n_k)}_{k \in \mathbb{N}}$ eine Folge in $\mathbb{N}$,
  welche streng monoton wachsend ist, das heisst
  \[
    n_0 < n_1 < n_2 < \cdots.
  \]
  Dann heisst die Folge ${(a_{n_k})}_{k \in \mathbb{N}}$ 
eine \emph{Teilfolge} von ${(a_n)}_{n \in \mathbb{N}}$.
\end{definition}

In anderen Worten ist eine Teilfolge einer Folge
die Folge die man erhält, wenn man Folgeglieder
wegnimmt, so dass aber noch unendlich viele
Folgeglieder übrigbleibt.

\begin{manualtheorem}{2}[Bolzano-Weierstrass]
  Sei ${(a_n)}_{n \in \mathbb{N}}$ eine Folge in $\mathbb{R}$,
  welche von oben und unten beschränkt ist.
  Dann hat die Folge ${(a_n)}_{n \in \mathbb{N}}$
  eine konvergente Teilfolge.
\end{manualtheorem}

\begin{example}
  Die Folge ${(a_n)}_{n \in \mathbb{N}}$ mit Folgegliedern
  $a_n = {(-1)}^n$ hat zwei naheliegende konvergente Teilfolgen,
  nämlich ${(a_{2k})}_{k \in \mathbb{N}}$ mit Grenzwert $1$,
  und ${(a_{2k+1})}_{k \in \mathbb{N}}$ mit Grenzwert $-1$.
  Es gibt aber noch mehr Folgen von Indizes 
  ${(n_k)}_{k \in \mathbb{N}}$, die
  konvergente Teilfolgen liefern.
\end{example}

\begin{proof}[Beweis von Theorem 2]
  Sei ${(a_n)}_{n \in \mathbb{N}}$ beschränkt, das heisst
  es existieren $x, y \in \mathbb{R}$, so dass für
  alle $n \in \mathbb{N}$ gilt, dass $x \leq a_k \leq y$.
  Wir unterscheiden nun zwei Fälle.
  \begin{enumerate}[1.]
    \item Die Menge $\{a_0, a_1, a_2, \dots\}$ ist endlich.
      Dann hat die Folge ${(a_n)}_{n \in \mathbb{N}}$ 
      eine konstante Teilfolge
      ${(a_{n_k})}_{k \in \mathbb{N}}$.
      Diese ist konvergent.
    \item Die Menge $\{a_0, a_1, a_2, \dots\}$ ist unendlich.
      Wir konstruieren eine sogenannte Intervallschachtelung.
      Setze 
      \[
        I_0 = [x, y] = \left\{z \in \mathbb{R} 
        \mid x \leq z \leq y\right\}.
      \]
      Wähle $a_{n_0} = a_0 \in I_0$. Halbiere $I_0$, also
      schreibe
      \[
        I_0 = \left[ x, \frac{x+y}{2} \right] 
        \cup \left[ \frac{x+y}{2}, y \right].
      \]
      Eines dieser beiden Teilintervalle enthält
      unendlich viele Folgeglieder,
      das heisst Elemente der Menge
      $\{a_0, a_1, a_2, \dots\}$. Nenne dieses Intervall
      $I_1 = [x_1, y_1]$. Wähle $a_{n_1} \in I_1$
      mit $n_1 > n_0$. Das können wir fordern,
      da es unendlich viele Folgeglieder in 
      $I_1$ gibt.
      Halbiere wie oben $I_1$ und schreibe
      \[
        I_1 = \left[ x_1, \frac{x_1 + y_1}{2} \right] 
        \cup \left[ \frac{x_1 + y_1}{2}, y_1 \right].
      \]
      Wieder enthält eines dieser Intervalle
      unendlich viele Folgeglieder. Nenne dieses
      Intervall $I_2 = [x_2, y_2]$. Wähle
      $a_{n_2} \in I_2$ mit $n_{2} > n_1$.
      Iteriere dieses Verfahren und erhalte eine
      Teilfolge $a_{n_0}, a_{n_1}, a_{n_2}, \dots$.

      Wir behaupten, dass ${(a_{n_k})}_{k \in \mathbb{N}}$
      eine Cauchyfolge (und deshalb konvergent) ist.
      Zum Beweis berechnen wir die Breite des
      Intervalls $I_k$. Berechne
      \[
        y_k - x_k = \frac{y - x}{2^k}.
      \]
      Sei $\varepsilon > 0$ vorgegeben. Wähle
      $N \in \mathbb{N}$, so dass
      \[
        \frac{y-x}{2^N} \leq \varepsilon.
      \]
      Dann gilt für alle $k, \ell \geq N$, dass
      $a_{n_k}, a_{n_\ell} \in I_N$ höchstens
      $\varepsilon$ voneinander entfernt sind.
      Konkreter,
      \[
        |a_{n_k} - a_{n_\ell}| 
        \leq \frac{y-x}{2^N} \leq \varepsilon. \qedhere
      \]
  \end{enumerate}
\end{proof}

Zusammengefasst haben wir die Intervalle
iterativ halbiert, und uns immer für das Intervall
entschieden, das immer noch unendlich viele
Folgeglieder enthalten hat.

\begin{definition}
  Eine Zahl $\alpha \in \mathbb{R}$ heisst
  \emph{Häufungspunkt} einer Folge
  ${(a_{n})}_{n \in \mathbb{N}}$ in $\mathbb{R}$,
  falls es eine konvergente Folge
  ${(a_{n_k})}_{k \in \mathbb{N}}$ gibt
  mit
  \[
    \lim_{k \to \infty} a_{n_k} = \alpha.
  \]
\end{definition}

\begin{example}
  \leavevmode
  \begin{enumerate}[(i)]
    \item Bei der Folge $a_n = (-1)^n$ ist die Menge
      der Häufungspunkte $\{-1, 1\}$.
    \item Sei $a \colon \mathbb{N} \to \mathbb{Q}$ eine
      beliebige
      Bijektion. Dann ist jede reelle Zahl ein Häufungspunkt,
      das heisst die Menge der Häufungspunkte
      ist $\mathbb{R}$. Das ist erstaunlich, da abzählbar
      viele Folgeglieder überabzählbar viele
      Häufungspunkte hervorrufen. Zum Beweis,
      dass tatsächlich jede reelle Zahl ein Häufungspunkt
      ist, sei $\alpha \in \mathbb{R}$. Wähle
      $n_0 \in \mathbb{N}$ mit $|\alpha - a_{n_0}| \leq 1$.
      Wähle dann $n_{1} > n_0$ mit $|\alpha - a_{n_1}| \leq 1/2$.
      Wähle iterativ $n_{k} > n_{k-1}$ mit $|\alpha - a_{n_k}|
      \leq 1/2^k$. Dies ist möglich, da jedes Intervall der
      Form
      $[\alpha - 1/2^k, \alpha + 1/2^k]$ unendlich viele
      rationale Zahlen enthält. Wir können auch $n_k > n_{k-1}$
      einrichten, da beider $k$-ten Wahl erst endlich viele
      rationale Zahlen bereits belegt sind.
  \end{enumerate}
\end{example}


\subsection*{Rechenregeln}
\begin{proposition}
  Seien ${(a_n)}_{n \in \mathbb{N}}$ und
  ${(b_n)}_{n \in \mathbb{N}}$ zwei
  konvergente Folgen in $\mathbb{R}$ mit Grenzwerten
  \[
    \lim_{n \to \infty} a_n = a \text{ und }
    \lim_{n \to \infty} b_n = b.
  \]
  Dann gilt:
  \begin{enumerate}[\normalfont(i)]
    \item $\lim_{n \to \infty} (a_n + b_n) = a + b$ 
    \item $\lim_{n \to \infty} (a_n \cdot b_n) = a \cdot b$
    \item Falls für alle $n \in \mathbb{N}$ gilt,
      dass $a_n \neq 0$ und auch $a \neq 0$ gilt,
      dann ist
      \[
        \lim_{n \to \infty} \frac{1}{a_n} = \frac{1}{a}.
      \]
  \end{enumerate}
\end{proposition}

\begin{remark}
Bei (iii) müssen wir tatsächlich annehmen,
dass $a \neq 0$. Die Folge ${(a_n)}_{n \in \mathbb{N}}$ 
mit Folgegliedern
\[
  a_n = \frac{1}{n+1}
\]
ist konvergent mit Grenzwert $a = 0$, aber die Folge
${(1/a_n)}_{n \in \mathbb{N}}$ konvergiert nicht in $\mathbb{R}$.
\end{remark}

\begin{proof}[Beweisskizze]
  \leavevmode
  \begin{enumerate}[(i)]
    \item $|(a_n + b_n) - (a + b)| \leq |a_n - a| + |b_n - b|$
    \item $|a_n b_n - ab| \leq 
      {|a_n b_n - a_n b + a_n b - ab |}
      \leq |a_n| |b_n - b| + |b||a_n -a|$. \qedhere
  \end{enumerate}
\end{proof}


\section{Reihen}
\begin{definition}
  Sei ${(a_{n})}_{n \in \mathbb{N}}$ eine Folge in $\mathbb{R}$.
  Dann heisst die formale Summe
  \[
    \sum_{k=0}^{\infty} a_k = a_0 + a_1 + a_2 + \cdots
  \]
  eine \emph{Reihe}. Eine Reihe
  $
    \sum_{k=0}^{\infty} a_k
  $
  heisst \emph{konvergent} mit Grenzwert $S \in \mathbb{R}$,
  falls die Folge ${(s_{n})}_{n \in \mathbb{N}}$ der
  \emph{Partialsummen}
  \[
    s_n = \sum_{k=0}^{n} a_k = a_0 + a_1 + \cdots + a_n
  \]
  in $\mathbb{R}$ mit Grenzwert $S$ konvergiert, das heisst
  \[
    \lim_{n \to \infty} s_n = S
  \]
  erfüllt. Wir schreiben in diesem Fall
  \[
    \sum_{k=0}^{\infty} a_k = S,
  \]
  was zwei Aussagen sind: dass die Folge ${(s_{n})}_{n \in \mathbb{N}}$ der Partialsummen konvergiert, und dass
  ihr Grenzwert $S$ ist.
\end{definition}

\subsection*{Die konstante Reihe}
Sei $c \in \mathbb{R}$ und setze $a_k = c$ für alle 
$k \in \mathbb{N}$.
Die zugehörige Reihe ist dann
 \[
  \sum_{k=0}^{\infty} c = c + c + c + \cdots.
\]
Wir berechnen, dass $s_n = (n+1)c$.
Falls $c \neq 0$ ist, dann ist die Folge
${(s_{n})}_{n \in \mathbb{N}}$ (nach dem
archimedischen Prinzip) unbeschränkt,
also nicht konvergent. Falls $c = 0$,
dann ist $s_n = 0$ für alle $n \in \mathbb{N}$,
und somit
\[
  \sum_{k=0}^{\infty} 0 = 0.
\]

\begin{remark}
  Sei $\sum_{k=0}^{\infty} a_k$ eine konvergente Reihe.
  Dann ist insbesondere ${(s_{n})}_{n \in \mathbb{N}}$ 
  eine Cauchyfolge.
  Für jedes $\varepsilon > 0$ existiert also ein Index
  $N \in \mathbb{N}$ mit der Eigenschaft, dass
  immer wenn $n, m \geq N$, dann ist
  \[
    |s_m - s_n| \leq \varepsilon.
  \]
  Insbesondere folgt für $m = n + 1$, dass
  \[
    |a_{n+1}| = |s_{n+1} - s_n| \leq \varepsilon.
  \]
  Wir folgern, dass
  \[
    \lim_{n \to \infty}a_n = 0.
  \]
  Diese Bedingung heisst  \emph{Cauchy-Bedingung}.
  Die Cauchy-Bedingung ist zwar notwendig, aber nicht
  hinreichend für Konvergenz einer Reihe.
\end{remark}

\subsection*{Die harmonische Reihe}
Aus der Divergenz der konstanten Reihe folgern wir
nun die Divergenz der harmonischen Reihe.
Sei $a_n = 1/n$ für $n \geq 1$. Die Reihe
$\sum_{k=1}^{\infty} 1/k$ heisst die
\emph{harmonische Reihe}, ausgeschrieben
\[
  \sum_{k=1}^{\infty} \frac{1}{k} = 
  1 + \frac{1}{2} + \frac{1}{3}
  + \frac{1}{4}
  + \frac{1}{5}
  + \frac{1}{6}
  + \frac{1}{7}
  + \frac{1}{8}
  + \cdots.
\]
Bemerke, dass wir die Terme der harmonischen Reihe
in Gruppen unterteilen können, deren Summe
jeweils grösser oder gleich $1/2$ ist. Nämlich ist
zum Beispiel
\begin{itemize}
  \item $1/2 \geq 1/2$,
  \item $1/3 + 1/4 \geq 2\cdot 1/4 = 1/2$,
  \item $1/5 + 1/6 + 1/7 + 1/8 \geq 4 \cdot 1/8 = 1/2$.
\end{itemize}
Allgemeiner gilt:
\[
  \frac{1}{n+1} + \frac{1}{n+2} + \cdots + \frac{1}{2n}
  \geq 2n \cdot \frac{1}{2n} = \frac{1}{2}.
\]
Also ist die Folge der Partialsummen $s_n$ nicht
beschränkt, und somit die 
harmonische Reihe 
$\sum_{k=0}^{\infty} 1/k$ nicht konvergent.

\subsection*{Die geometrische Reihe}
Sei $q \in \mathbb{R}$. Die Reihe
\[
  \sum_{k=0}^{\infty} q^k = 1 + q + q^2 + \cdots
\]
heisst \emph{geometrische Reihe}.


\begin{claim}
  Die Folge
  $\sum_{k=0}^{\infty}$ konvergent ist, genau dann,
  wenn $|q| < 1$, und in diesem Fall gilt
  \[
    \sum_{k=0}^{\infty} q^k = \frac{1}{1-k}
  \]
\end{claim}

\begin{proof}
  \leavevmode
  \begin{itemize}
    \item Falls $|q| \geq 1$, dann ist $0$ nicht
      der Grenzwert der Folge ${(q^n)}_{n \in \mathbb{N}}$ 
      Nach der Cauchy-Bedingung ist die Reihe
      $\sum_{k=0}^{\infty} q^k$ nicht konvergent.
    \item Falls $|q|<1$, dann gilt
      $\lim_{n \to \infty} q^n = 0$. Wir berechnen
      die Partialsummen
      \[
        s_n  \sum_{k=0}^{k} q^k 
            = 1 + q + q^2 + \cdots + q^n
      \]
      indem wir bemerken, dass
      \[
        q \cdot s_n - s_n = q^{n+1} - 1.
      \]
      Lösen wir nach $s_n$ auf, erhalten wir
      \[
        s_n = \frac{q^{n+1} - 1}{q - 1}
        = \frac{q^{n+1}}{q-1} - \frac{1}{q-1},
      \]
      und somit
      \[
        \lim_{n \to \infty} s_n = \frac{1}{1-q}. \qedhere
      \]
  \end{itemize}
\end{proof}

\subsection*{Folgerungen aus der Konvergenz der
geometrischen Reihe}
Aus der Konvergenz der geometrischen Reihe
$\sum_{k=0}^{\infty} q^k$ für $|q|<1$
können wir Konvergenz mehrerer interessanter
Reihen herleiten.
\begin{enumerate}[(1)]
  \item Sei $a_n = 1/n^2$ für $n \geq 1$. Wir untersuchen die Reihe
    \[
      \sum_{k=1}^{\infty} \frac{1}{k^2}
    \]
    indem wir die Partialsummen folgendermassen abschätzen.
    Zum Beispiel ist
    \begin{itemize}
      \item $1/2^2 + 1/3^2 < 2 \cdot 1/2^2$,
      \item $1/4^2 + 1/5^2 + 1/6^2 + 1/7^2 < 4 \cdot 1/4^2$,
      \item $1/8^2 + \cdots + 1/15^2 < 8 \cdot 1/8^2$.
    \end{itemize}
    Es gilt also für alle $n \in \mathbb{N}$, dass
    \[
      s_n \leq 1 + \frac{1}{2} + \frac{1}{4} + \frac{1}{8}
      \leq \sum_{k=1}^{\infty} {\left( \frac{1}{2} \right)}^k
      = \frac{1}{1-1/2} = 2.
    \]
    Die Folge ${(s_{n})}_{n \in \mathbb{N}}$ ist also
    monoton wachsend und nach oben beschränkt, also
    konvergent. Also ist
    \[
      \sum_{k=1}^{\infty} \frac{1}{k^2} = S
    \]
    mit $S \leq 2$. Euler hat gezeigt, dass $S = \pi^2/6$.
  \item Sei $a_n = 1/n^{1 + \alpha}$ mit $\alpha > 0$.
    Schätze die Partialsummen folgendermassen ab:
    \begin{itemize}
      \item $1/2^{1 + \alpha} + 
        1/3^{1 + \alpha}< 2 \cdot 2 \cdot 1/2^{1+\alpha}
        = 1/2^\alpha$,
      \item $1/4^{1 + \alpha} + \cdots + 1/7^{1+ \alpha}
        < 4 \cdot 1/4^{1 + \alpha} = 1/4^\alpha$,
      \item $1/8^{1 + \alpha} + \cdots + 1/15^{1 + \alpha}
        < 8 \cdot 1 /8^{1 + \alpha} = 1/8^{\alpha}$.
    \end{itemize}
    Für alle $n \in \mathbb{N}$ gilt also, dass
    \[
      s_n \leq 1 + \frac{1}{2^{\alpha}} + \frac{1}{4^{\alpha}}
      + \frac{1}{8^{\alpha}} + \dots
      = \sum_{k=0}^{\infty} 
      {\left( \frac{1}{2^{\alpha}} \right)}^k
      = \frac{1}{1-1/2^{\alpha}},
    \]
    wobei wir die Konvergenz der geometrischen Reihe
    $\sum_{k=0}^{\infty} q^k$ für $q = 1/2^\alpha < 1$
    angewendet haben.
  \item Sei $a_n = 1/n!$ (mit der Konvention $0! = 1$).
    Die Reihe $\sum_{k=0}^{\infty} 1/k!$ kann man
    termweise durch eine geometrische Reihe abschätzen:
    \[
      \sum_{k=0}^{\infty} \frac{1}{k!} = \frac{1}{0!}
      + \frac{1}{1!}
      + \frac{1}{2!}
      + \frac{1}{3!}
      + \frac{1}{4!}
      + \frac{1}{5!}
      \leq
      1 + 1 + \frac{1}{2} + \frac{1}{4} + \frac{1}{8}
      + \frac{1}{16} + \cdots = 3.
    \]
    Die Details hier sind eine Übung auf Serie 4.
    Wir schliessen, dass die monoton wachsende Folge
    \[
      s_n = \sum_{k=0}^{n} \frac{1}{k!}
    \]
    konvergiert, mit Grenzwert $S \leq 3$.
\end{enumerate}

\begin{remark}
  Der Wert der obigen Reihe ist $e$, die Eulersche Zahl:
  \[
    \sum_{k=0}^{\infty} \frac{1}{k!} = \lim_{n \to \infty}
    {\left( 1+ \frac{1}{n} \right)}^n = e.
  \]
  Wir zeigen die erste dieser Gleichungen später.
\end{remark}

\begin{summary}
  Alle allgemein anwendbaren Verfahren zur Untersuchung
  der Konvergenz von Reihen von positiven
  Termen basieren auf folgenden
  beiden Strategien.
  \begin{itemize}
    \item Um die Konvergenz einer Reihe
      $\sum_{k=0}^{\infty} a_k$ (mit $a_k > 0$) zu zeigen,
      schätzen wir diese nach oben durch
      eine geometrische Reihe ab.
    \item Um die Divergenz (nicht-Konvergenz) einer
      solchen Reihe zu zeigen, schätzen wir diese nach unten
      durch eine konstante Reihe ab.
  \end{itemize}
\end{summary}

\begin{question}
  Was, wenn die Reihe $\sum_{k=0}^{\infty} a_k$ auch
  negative Summanden $a_k < 0$ enthält?
\end{question}

\begin{example}
  Sei $a_n = {(-1)}^n \cdot 1/(n+1)$.
  Wir fragen uns, ob die Reihe
  \[
    \sum_{k=0}^{\infty} {(-1)}^k \frac{1}{k+1}
  \]
  konvergiert. Unsere obigen Verfahren scheinen 
  nicht zu funktionieren (ausser man hat eine
  gute Idee).
\end{example}

\begin{claim}[Leibnizkriterium]
  Sei ${(a_{n})}_{n \in \mathbb{N}}$ eine Folge
  positiver reller Zahlen mit
  $a_0 > a_1 > a_2 > \cdots > 0$, eine sogenannte
  monoton fallende Folge, mit 
  \[
    \lim_{n \to \infty}a_n = 0.
  \]
  Dann konvergiert die alternierende Reihe
  \[
    \sum_{k=0}^{\infty} {(-1)}^k a_k.
  \]
\end{claim}

\begin{proof}
  Betrachte die Partialsummen $s_n = \sum_{k=0}^{n} {(-1)}^k a_k$.
  Wir stellen fest:
  \begin{enumerate}[(i)]
    \item Wir haben $s_0 \geq s_2 \geq s_4 \geq s_6 \geq \cdots$,
      da $s_{k+2} = s_{2k} - a_{2k+1} + a_{2k+2}$ und
      $a_{2k+2}<a_{2k+1}$.
    \item Analog haben wir 
      $s_1 \leq s_3 \leq s_5 \leq s_7 \leq \cdots$ (die
      Ungleichungen gehen in die andere Richtung).
    \item Ausserdem gilt für alle $k \in \mathbb{N}$, dass
      $s_{2k+1} \leq s_{2k+2}$.
  \end{enumerate}
  Aus diesen drei Beobachtungen folgt, dass
  \[
    s_1 \leq s_3 \leq s_5 \leq \cdots \leq s_6 \leq s_4 \leq
    s_2 \leq s_0.
  \]
  Sei nun $\varepsilon > 0$ vorgegeben. Wähle 
  eine ungerade Zahl $N \in \mathbb{N}$ 
  mit der Eigenschaft, dass wenn immer $n \geq N$, dann
  ist $|a_n| \leq \varepsilon$. Solch ein $N$ existiert,
  da wir angenommen haben, dass
  \[
    \lim_{n \to \infty} a_n = 0
  \]
  (ersetze wenn nötig $N$ durch $N + 1$).
  Für alle $n \geq N$ gilt dann
  \[
    s_N \leq s_n \leq s_{N+1}.
  \]
  Ebenso gilt für alle $m \geq N$, dass
  $s_N \leq s_m \leq s_{N+1}$. Dann ist aber
  \[
    |s_n - s_m| \leq |s_N - s_{N+1}| = |a_{N+1}| \leq \varepsilon.
  \]
  Also ist die Folge ${(s_{n})}_{n \in \mathbb{N}}$
  eine Cauchyfolge in $\mathbb{R}$ und somit konvergent.
\end{proof}




\end{document}

